\documentclass[english,14pt,dvipsnames,table,fleqn]{beamer}

\usepackage{amsmath,amsfonts,amssymb}
\usepackage{appendixnumberbeamer}
\usepackage{array}
\usepackage{babel}
\usepackage[scaled=0.81]{beramono}
\usepackage{booktabs}
\usepackage[T1]{fontenc}
\usepackage{graphicx}
\usepackage[utf8]{inputenc}
\usepackage{lmodern}
\usepackage{mathtools}
\usepackage{minibox}
\usepackage{minted}  % Syntax highlighting using pygmentize
\usepackage{multicol}
\usepackage{pbox}
\usepackage{tabularx}
\usepackage{times}
\usepackage[normalem]{ulem}
\usepackage{wasysym}
\usepackage{xparse}
\usepackage{xspace}
\usepackage[utf8]{inputenc}
\usepackage{agda}
\usepackage{etoolbox}

\usepackage{minted}
\newminted[Haskell]{haskell}{mathescape,escapeinside=~~}
\newmintinline[Hs]{haskell}{}
\definecolor{markinlineColor}{RGB}{204,229,255}
\newmintinline[Hc]{haskell}{bgcolor=markinlineColor}


\newminted[Java]{java}{mathescape,escapeinside=~~}

\newminted[JavaScript]{javascript}{mathescape,escapeinside=~~}


\usepackage{color,colortbl,xcolor}
\usepackage{comment}

\usepackage{tikz}
\usetikzlibrary{arrows,arrows.meta}
\usetikzlibrary{calc}
\usetikzlibrary{fit}
\usetikzlibrary{overlay-beamer-styles}
\usetikzlibrary{positioning}
\usetikzlibrary{shapes,shapes.symbols}
\usetikzlibrary{tikzmark}
\usetikzlibrary{trees}

\definecolor{mc1}{RGB}{27,158,119}
\definecolor{mc2}{RGB}{217,95,2}
\definecolor{mc3}{RGB}{117,112,179}

\usepackage{pgfplots}
\usepgfplotslibrary{dateplot}

\usepackage[absolute,overlay]{textpos}  % place images with absolute position
  \setlength{\TPHorizModule}{1mm}
  \setlength{\TPVertModule}{1mm}

% Use metropolis beamer theme
\usetheme{metropolis}
\usepackage{FiraSans}  % Really nice font for metro theme
\setbeamercolor{progress bar}{fg=darkgray}

\setbeamertemplate{itemize item}{$\bullet$}
\setbeamertemplate{navigation symbols}{}
\setbeamertemplate{footline}[frame number]

% Define block styles
\tikzstyle{decision} = [diamond, draw, fill=blue!20, text width=4.5em,
  text badly centered, node distance=3cm, inner sep=0pt]
\tikzstyle{block} = [rectangle, draw, fill=mDarkTeal, font=\footnotesize,
  text=white, text width=4em, text centered, rounded corners, minimum height=3em]
\tikzstyle{line} = [draw, -latex']
\tikzstyle{cloud} = [draw, ellipse,fill=mDarkTeal!20, node distance=2.5cm,
  font=\footnotesize, minimum height=2em]

% Stuff for the title page
\title{\lecturetitle}
\author{Jesper Cockx}
\institute{Technical University Delft}
%\titlegraphic{\includegraphics[height=2cm]{TUDelft.png}}


% `to' (destination) node in code
\def\soToNode#1{\node (#1) at ($(pic cs:#1) + (-3pt,0)$) {}}
% `from' (comment) node
\DeclareDocumentCommand{\soFromNode}{ m m m }{\node (_#1) [#3 #1] {#2}}
% arrow connecting `from' and `to'
\DeclareDocumentCommand{\soArrow}{ m m }{\draw [#2] (_#1) -> (#1)}
% #1             #2                   #3                 #4                #5
% tikzmark name, text in description, where description, arrow attributes, arguments to tikzpicture
\DeclareDocumentCommand{\soComment}{ m m O{below=4mm of} O{black,very thick} O{} }{%
\begin{tikzpicture}[remember picture, ->, >=stealth, overlay, red, very thick, align=left,#5]
  \soToNode{#1};
  \soFromNode{#1}{#2}{#3};
  \soArrow{#1}{#4};
\end{tikzpicture}
}

\tikzset{var/.style={draw,rectangle,blue,very thick,minimum size=30}}
\tikzset{obj/.style={draw,ellipse,minimum width=30,thick}}

\tikzset{cls/.style={draw,ellipse,minimum width=70,very thick}}
\tikzset{int/.style={draw,ellipse,minimum width=70,very thick, dotted}}

% http://tex.stackexchange.com/questions/100313/a-problem-with-beamer-tikz-and-fonts
\let\next\relax  % See URL above for why this is needed with TikZ + Beamer


\setbeamertemplate{section in toc}[square]

\makeatletter
\patchcmd{\beamer@sectionintoc}{\vskip1.5em}{\vskip0.5em}{}{}
\makeatother

\AtBeginSection[]
{
  \begin{frame}<beamer>[plain,noframenumbering]{Outline}
    \tableofcontents[currentsection,hideallsubsections]
  \end{frame}
}

\makeatletter
\newlength\beamerleftmargin
\setlength\beamerleftmargin{\Gm@lmargin}
\makeatother

\renewcommand{\arraycolsep}{3pt}

\newdimen\mathindent
\mathindent=2ex


\usepackage{newunicodechar}
\newunicodechar{→}{\ensuremath{\rightarrow}}
\newunicodechar{←}{\ensuremath{\leftarrow}}
\newunicodechar{×}{\ensuremath{\times}}
\newunicodechar{λ}{\ensuremath{\lambda}}
\newunicodechar{∀}{\ensuremath{\forall}}
\newunicodechar{Π}{\ensuremath{\Pi}}
\newunicodechar{Σ}{\ensuremath{\Sigma}}
\newunicodechar{≡}{\ensuremath{\equiv}}
\newunicodechar{≅}{\ensuremath{\cong}}
\newunicodechar{≐}{\ensuremath{\doteq}}
\newunicodechar{∈}{\ensuremath{\in}}
\newunicodechar{∧}{\ensuremath{\land}}
\newunicodechar{∨}{\ensuremath{\lor}}
\newunicodechar{⊤}{\ensuremath{\top}}
\newunicodechar{⊥}{\ensuremath{\bot}}
\newunicodechar{⊔}{\ensuremath{\sqcup}}
\newunicodechar{∷}{\ensuremath{{::}}}
\newunicodechar{ℓ}{\ensuremath{\ell}}
\newunicodechar{₀}{\ensuremath{{_0}}}
\newunicodechar{₁}{\ensuremath{{_1}}}
\newunicodechar{₂}{\ensuremath{{_2}}}
\newunicodechar{₃}{\ensuremath{{_3}}}
\newunicodechar{₄}{\ensuremath{{_4}}}
\newunicodechar{₅}{\ensuremath{{_5}}}
\newunicodechar{₆}{\ensuremath{{_6}}}
\newunicodechar{₇}{\ensuremath{{_7}}}
\newunicodechar{₈}{\ensuremath{{_8}}}
\newunicodechar{₉}{\ensuremath{{_9}}}
\newunicodechar{⟨}{\ensuremath{{\langle}}}
\newunicodechar{⟩}{\ensuremath{{\rangle}}}
\newunicodechar{̧}{\c}
\newunicodechar{∘}{\ensuremath{\circ}}
\newunicodechar{≤}{\ensuremath{\leq}}
\newunicodechar{⋯}{\ensuremath{\cdots}}
\newunicodechar{ℕ}{\ensuremath{\mathbb{N}}}
\newunicodechar{∎}{\ensuremath{\blacksquare}}
\newunicodechar{∞}{\ensuremath{\infty}}
\newunicodechar{⊎}{\ensuremath{\uplus}}
\newunicodechar{¬}{\ensuremath{\lnot}}
\newunicodechar{⇓}{\ensuremath{\Downarrow}}
\newunicodechar{≟}{\ensuremath{\stackrel{?}{=}}}
\newunicodechar{ν}{\ensuremath{\nu}}
\newunicodechar{δ}{\ensuremath{\delta}}
\newunicodechar{∅}{\ensuremath{\emptyset}}
\newunicodechar{ε}{\ensuremath{\varepsilon}}
\newunicodechar{∋}{\ensuremath{\ni}}
\newunicodechar{∩}{\ensuremath{\cap}}
\newunicodechar{∪}{\ensuremath{\cup}}
\newunicodechar{·}{\ensuremath{\cdot}}

\newcommand\var[1]{\mathit{#1}}
\newcommand\prim[1]{{\AgdaPrimitive{#1}}}
\newcommand\ty[1]{{{\prim{Set}}_{#1}}}
\newcommand\fun[1]{{\AgdaFunction{#1}}}
\newcommand\data[1]{{\AgdaFunction{#1}}}
\newcommand\con[1]{{\AgdaInductiveConstructor{#1}}}
\newcommand\field[1]{{\AgdaField{#1}}}
\newcommand\keyw[1]{{\AgdaKeyword{#1}}}
\newcommand\lit[1]{{\AgdaNumber{#1}}}
\newcommand\level{\fun{Level}}
\newcommand\ite[3]{\fun{if}\  #1\  \fun{then}\  #2\  \fun{else}\  #3}


\newcommand\Nat{\data{ℕ}}
\newcommand\zero{\con{zero}}
\newcommand\suc{\con{suc}}
\newcommand\Bool{\data{Bool}}
\newcommand\true{\con{true}}
\newcommand\false{\con{false}}
\newcommand\List{\data{List}}
\renewcommand\Vec{\data{Vec}}
\newcommand\nil{\con{[]}}
\newcommand\cons{\con{::}}
\newcommand\Fin{\data{Fin}}
\renewcommand\prod{\data{×}}
\newcommand\sigmatype{\data{Σ}}
\newcommand\toptype{\data{⊤}}
\newcommand\bottomtype{\data{⊥}}
\newcommand\Id{\data{≡}}
\newcommand\refl{\con{refl}}

\newcommand\prf[1]{\colorbox{gray!30}{#1}}
\newcommand{\ru}[2]{\dfrac{\begin{array}{@{}c@{}}#1\end{array}}{\begin{array}{@{}c@{}}#2\end{array}}}



%%% Local Variables:
%%% mode: latex -shell-escape
%%% TeX-master: t
%%% End:
