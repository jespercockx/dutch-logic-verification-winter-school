\documentclass[english,14pt,dvipsnames,table,fleqn]{beamer}

\usepackage{amsmath,amsfonts,amssymb}
\usepackage{appendixnumberbeamer}
\usepackage{array}
\usepackage{babel}
\usepackage[scaled=0.81]{beramono}
\usepackage{booktabs}
\usepackage[T1]{fontenc}
\usepackage{graphicx}
\usepackage[utf8]{inputenc}
\usepackage{lmodern}
\usepackage{mathtools}
\usepackage{minibox}
\usepackage{minted}  % Syntax highlighting using pygmentize
\usepackage{multicol}
\usepackage{pbox}
\usepackage{tabularx}
\usepackage{times}
\usepackage[normalem]{ulem}
\usepackage{wasysym}
\usepackage{xparse}
\usepackage{xspace}
\usepackage[utf8]{inputenc}
\usepackage{agda}
\usepackage{etoolbox}

\usepackage{minted}
\newminted[Haskell]{haskell}{mathescape,escapeinside=~~}
\newmintinline[Hs]{haskell}{}
\definecolor{markinlineColor}{RGB}{204,229,255}
\newmintinline[Hc]{haskell}{bgcolor=markinlineColor}


\newminted[Java]{java}{mathescape,escapeinside=~~}

\newminted[JavaScript]{javascript}{mathescape,escapeinside=~~}


\usepackage{color,colortbl,xcolor}
\usepackage{comment}

\usepackage{tikz}
\usetikzlibrary{arrows,arrows.meta}
\usetikzlibrary{calc}
\usetikzlibrary{fit}
\usetikzlibrary{overlay-beamer-styles}
\usetikzlibrary{positioning}
\usetikzlibrary{shapes,shapes.symbols}
\usetikzlibrary{tikzmark}
\usetikzlibrary{trees}

\definecolor{mc1}{RGB}{27,158,119}
\definecolor{mc2}{RGB}{217,95,2}
\definecolor{mc3}{RGB}{117,112,179}

\usepackage{pgfplots}
\usepgfplotslibrary{dateplot}

\usepackage[absolute,overlay]{textpos}  % place images with absolute position
  \setlength{\TPHorizModule}{1mm}
  \setlength{\TPVertModule}{1mm}

% Use metropolis beamer theme
\usetheme{metropolis}
\usepackage{FiraSans}  % Really nice font for metro theme
\setbeamercolor{progress bar}{fg=darkgray}

\setbeamertemplate{itemize item}{$\bullet$}
\setbeamertemplate{navigation symbols}{}
\setbeamertemplate{footline}[frame number]

% Define block styles
\tikzstyle{decision} = [diamond, draw, fill=blue!20, text width=4.5em,
  text badly centered, node distance=3cm, inner sep=0pt]
\tikzstyle{block} = [rectangle, draw, fill=mDarkTeal, font=\footnotesize,
  text=white, text width=4em, text centered, rounded corners, minimum height=3em]
\tikzstyle{line} = [draw, -latex']
\tikzstyle{cloud} = [draw, ellipse,fill=mDarkTeal!20, node distance=2.5cm,
  font=\footnotesize, minimum height=2em]

% Stuff for the title page
\title{\lecturetitle}
\author{Jesper Cockx}
\institute{Technical University Delft}
%\titlegraphic{\includegraphics[height=2cm]{TUDelft.png}}


% `to' (destination) node in code
\def\soToNode#1{\node (#1) at ($(pic cs:#1) + (-3pt,0)$) {}}
% `from' (comment) node
\DeclareDocumentCommand{\soFromNode}{ m m m }{\node (_#1) [#3 #1] {#2}}
% arrow connecting `from' and `to'
\DeclareDocumentCommand{\soArrow}{ m m }{\draw [#2] (_#1) -> (#1)}
% #1             #2                   #3                 #4                #5
% tikzmark name, text in description, where description, arrow attributes, arguments to tikzpicture
\DeclareDocumentCommand{\soComment}{ m m O{below=4mm of} O{black,very thick} O{} }{%
\begin{tikzpicture}[remember picture, ->, >=stealth, overlay, red, very thick, align=left,#5]
  \soToNode{#1};
  \soFromNode{#1}{#2}{#3};
  \soArrow{#1}{#4};
\end{tikzpicture}
}

\tikzset{var/.style={draw,rectangle,blue,very thick,minimum size=30}}
\tikzset{obj/.style={draw,ellipse,minimum width=30,thick}}

\tikzset{cls/.style={draw,ellipse,minimum width=70,very thick}}
\tikzset{int/.style={draw,ellipse,minimum width=70,very thick, dotted}}

% http://tex.stackexchange.com/questions/100313/a-problem-with-beamer-tikz-and-fonts
\let\next\relax  % See URL above for why this is needed with TikZ + Beamer


\setbeamertemplate{section in toc}[square]

\makeatletter
\patchcmd{\beamer@sectionintoc}{\vskip1.5em}{\vskip0.5em}{}{}
\makeatother

\AtBeginSection[]
{
  \begin{frame}<beamer>[plain,noframenumbering]{Outline}
    \tableofcontents[currentsection,hideallsubsections]
  \end{frame}
}

\makeatletter
\newlength\beamerleftmargin
\setlength\beamerleftmargin{\Gm@lmargin}
\makeatother

\renewcommand{\arraycolsep}{3pt}

\newdimen\mathindent
\mathindent=2ex


\usepackage{newunicodechar}
\newunicodechar{→}{\ensuremath{\rightarrow}}
\newunicodechar{←}{\ensuremath{\leftarrow}}
\newunicodechar{×}{\ensuremath{\times}}
\newunicodechar{λ}{\ensuremath{\lambda}}
\newunicodechar{∀}{\ensuremath{\forall}}
\newunicodechar{Π}{\ensuremath{\Pi}}
\newunicodechar{Σ}{\ensuremath{\Sigma}}
\newunicodechar{≡}{\ensuremath{\equiv}}
\newunicodechar{≅}{\ensuremath{\cong}}
\newunicodechar{≐}{\ensuremath{\doteq}}
\newunicodechar{∈}{\ensuremath{\in}}
\newunicodechar{∧}{\ensuremath{\land}}
\newunicodechar{∨}{\ensuremath{\lor}}
\newunicodechar{⊤}{\ensuremath{\top}}
\newunicodechar{⊥}{\ensuremath{\bot}}
\newunicodechar{⊔}{\ensuremath{\sqcup}}
\newunicodechar{∷}{\ensuremath{{::}}}
\newunicodechar{ℓ}{\ensuremath{\ell}}
\newunicodechar{₀}{\ensuremath{{_0}}}
\newunicodechar{₁}{\ensuremath{{_1}}}
\newunicodechar{₂}{\ensuremath{{_2}}}
\newunicodechar{₃}{\ensuremath{{_3}}}
\newunicodechar{₄}{\ensuremath{{_4}}}
\newunicodechar{₅}{\ensuremath{{_5}}}
\newunicodechar{₆}{\ensuremath{{_6}}}
\newunicodechar{₇}{\ensuremath{{_7}}}
\newunicodechar{₈}{\ensuremath{{_8}}}
\newunicodechar{₉}{\ensuremath{{_9}}}
\newunicodechar{⟨}{\ensuremath{{\langle}}}
\newunicodechar{⟩}{\ensuremath{{\rangle}}}
\newunicodechar{̧}{\c}
\newunicodechar{∘}{\ensuremath{\circ}}
\newunicodechar{≤}{\ensuremath{\leq}}
\newunicodechar{⋯}{\ensuremath{\cdots}}
\newunicodechar{ℕ}{\ensuremath{\mathbb{N}}}
\newunicodechar{∎}{\ensuremath{\blacksquare}}
\newunicodechar{∞}{\ensuremath{\infty}}
\newunicodechar{⊎}{\ensuremath{\uplus}}
\newunicodechar{¬}{\ensuremath{\lnot}}
\newunicodechar{⇓}{\ensuremath{\Downarrow}}
\newunicodechar{≟}{\ensuremath{\stackrel{?}{=}}}
\newunicodechar{ν}{\ensuremath{\nu}}
\newunicodechar{δ}{\ensuremath{\delta}}
\newunicodechar{∅}{\ensuremath{\emptyset}}
\newunicodechar{ε}{\ensuremath{\varepsilon}}
\newunicodechar{∋}{\ensuremath{\ni}}
\newunicodechar{∩}{\ensuremath{\cap}}
\newunicodechar{∪}{\ensuremath{\cup}}
\newunicodechar{·}{\ensuremath{\cdot}}

\newcommand\var[1]{\mathit{#1}}
\newcommand\prim[1]{{\AgdaPrimitive{#1}}}
\newcommand\ty[1]{{{\prim{Set}}_{#1}}}
\newcommand\fun[1]{{\AgdaFunction{#1}}}
\newcommand\data[1]{{\AgdaFunction{#1}}}
\newcommand\con[1]{{\AgdaInductiveConstructor{#1}}}
\newcommand\field[1]{{\AgdaField{#1}}}
\newcommand\keyw[1]{{\AgdaKeyword{#1}}}
\newcommand\lit[1]{{\AgdaNumber{#1}}}
\newcommand\level{\fun{Level}}
\newcommand\ite[3]{\fun{if}\  #1\  \fun{then}\  #2\  \fun{else}\  #3}


\newcommand\Nat{\data{ℕ}}
\newcommand\zero{\con{zero}}
\newcommand\suc{\con{suc}}
\newcommand\Bool{\data{Bool}}
\newcommand\true{\con{true}}
\newcommand\false{\con{false}}
\newcommand\List{\data{List}}
\renewcommand\Vec{\data{Vec}}
\newcommand\nil{\con{[]}}
\newcommand\cons{\con{::}}
\newcommand\Fin{\data{Fin}}
\renewcommand\prod{\data{×}}
\newcommand\sigmatype{\data{Σ}}
\newcommand\toptype{\data{⊤}}
\newcommand\bottomtype{\data{⊥}}
\newcommand\Id{\data{≡}}
\newcommand\refl{\con{refl}}

\newcommand\prf[1]{\colorbox{gray!30}{#1}}
\newcommand{\ru}[2]{\dfrac{\begin{array}{@{}c@{}}#1\end{array}}{\begin{array}{@{}c@{}}#2\end{array}}}



%%% Local Variables:
%%% mode: latex -shell-escape
%%% TeX-master: t
%%% End:


\def\lecturetitle{Coinductive Programming and Proving in Agda}
\newcommand\courseyear{2026}
\subtitle{Lecture 3: Coinduction case studies}
\date{21 January 2026}

\begin{document}

\maketitle

\section{The delay monad}

\begin{frame}[plain,noframenumbering]{Lecture plan}
  \tableofcontents[hideallsubsections]
\end{frame}

\begin{code}[hide]
{-# OPTIONS --allow-unsolved-metas --guardedness --sized-types --cubical -WnoUnsupportedIndexedMatch #-}
open import lec1-slides hiding (Bool; true; false; if_then_else_; not)
open import lec2-slides

open import Data.Bool.Base using (Bool; true; false; not; _∧_; _∨_; if_then_else_)

module _ where

module GuardedDelay where
  open Guardedness
  open import Relation.Binary.PropositionalEquality.Core
\end{code}

\begin{frame}{The delay monad}
The \alert{delay monad} embeds potentially non-terminating computations in Agda:
\begin{code}
  mutual
    data Delay (A : Set) : Set where
      now    : A → Delay A
      later  : Delay' A → Delay A

    record Delay' (A : Set) : Set where
      coinductive
      field force : Delay A
  open Delay' public
\end{code}
\end{frame}

\begin{frame}{Delayed values}
Classically, any $x : \data{Delay}~A$ is either $\fun{never}$ \\ or $\fun{laters}~k~a$ for some $k : \Nat$ and $a : A$:
\begin{code}
  never : Delay A
  never = later (λ where .force → never)

  laters : ℕ → A → Delay A
  laters zero a = now a
  laters (suc n) a = later
    (λ where .force → laters n a)
\end{code}
\textbf{Exercise.} Implement $\fun{iter} : (A → A~⊎~B) → A → \data{Delay}~B$.
\end{frame}

\begin{frame}{Implementing bind for the delay monad}
\begin{code}
  _>>=_ : Delay A → (A → Delay B) → Delay B
  now x >>= f = f x
  later d >>= f = later λ where
    .force → d .force >>= f
\end{code}
\end{frame}


\begin{frame}{Convergence of delayed values}
\begin{code}[hide]
  infix 3 _⇓_
  variable d d₁ d₂ : A
\end{code}
\begin{code}
  data _⇓_ {A} : Delay A → A → Set where
    now : (x : A) → now x ⇓ x
    later : d .force ⇓ x → later d ⇓ x

  ⇓-unique : d ⇓ x → d ⇓ y → x ≡ y
  ⇓-unique (now x) (now y) = refl
  ⇓-unique (later p) (later q) = ⇓-unique p q
\end{code}
\end{frame}


\begin{frame}{Bisimulation of delayed values}
\begin{code}
  mutual
    data _~_ {A} : Delay A → Delay A → Set where
      now    : (x : A)  → now    x  ~ now    x
      later  : x ~' y   → later  x  ~ later  y

    record _~'_ (x y : Delay' A) : Set where
      coinductive
      field
        force : x .force ~ y .force
  open _~'_ public
\end{code}
\begin{code}[hide]
  infix 5 _~_ _~'_
\end{code}
\end{frame}

\begin{frame}{Monad laws for Delay}
\begin{code}
  refl~ : (x : Delay A) → x ~ x
  refl~ (now x) = now x
  refl~ (later x) = later λ where
    .force → refl~ (x .force)

  now->>=  : (x : A) (f : A → Delay B)
           → now x >>= f ~ f x
  now->>= x f = refl~ (f x)
\end{code}
\textbf{Exercise.} State and prove the second and third monad laws \fun{>>=-now} and \fun{>>=-assoc}.
\begin{code}[hide]
  >>=-now : (d : Delay A) → d >>= now ~ d
  >>=-now (now x) = now x
  >>=-now (later x) = later λ where
    .force → >>=-now (x .force)

  >>=-assoc : (d : Delay A)
    → (f : A → Delay B) (g : B → Delay C)
    → (d >>= f) >>= g ~ d >>= λ x → (f x >>= g)
  >>=-assoc (now x) f g = refl~ (f x >>= g)
  >>=-assoc (later d) f g = later λ where
    .force → >>=-assoc (d .force) f g
\end{code}
\end{frame}


\begin{frame}{Weak bisimilarity}
Since we don't really care about the number of \con{later}s, we can make more things bisimilar:
\begin{code}
  mutual
    data _~D_ {A} : Delay A → Delay A → Set where
      value  : d₁ ⇓ x → d₂ ⇓ x → d₁ ~D d₂
      later  : d₁ ~D' d₂ → later d₁ ~D later d₂

    record _~D'_ (x y : Delay' A) : Set where
      coinductive
      field force : x .force ~D y .force
  open _~D'_ public
\end{code}
\end{frame}

\begin{frame}{The partiality monad}
Instead of making more elements bisimilar, we can quotient the \data{Delay} monad by $x ~ \fun{later}~x$.

This leads to the definition of the \alert{partiality monad} by Altenkirch, Danielsson \& Kraus (FoSSaC 2017)
as a \emph{quotient inductive-inductive type} (QIIT).
\end{frame}

\section{Stream processors}

\begin{code}[hide]
module StreamProcessors where
  open Lists using (List; []; _∷_)
  open Guardedness
  open InductiveIdentity
  open GuardedBisimulation
\end{code}

\begin{frame}{Stream processors}
A \alert{stream processor} describes how to transform a stream of $A$s into a stream of $B$s:
\small \begin{code}
  mutual
    data SP (A B : Set) : Set where
      get  : (A → SP A B)  → SP A B
      put  : B → SP' A B   → SP A B

    record SP' (A B : Set) : Set where
      coinductive
      field force : SP A B
  open SP'
\end{code}
\normalsize There can only be a finite number of \con{get}s before there must be a \con{put}.
\end{frame}

\begin{frame}{Example stream processor: summing elements pairwise}
\begin{code}
  sum2by2 : SP ℕ ℕ
  sum2by2 =
    get λ x →
    get λ y →
    put (x + y)
    λ where .force → sum2by2
\end{code}
\end{frame}

\begin{frame}{Running a stream processor}
\begin{code}
  run : SP A B → Stream A → Stream B
  run (get f) xs = run (f (xs .head)) (xs .tail)
  run (put y sp)  xs .head  = y
  run (put y sp)  xs .tail  = run (sp .force) xs

  sum2by2-nats :
    take 5 (run sum2by2 nats)
    ≡ (1 ∷ 5 ∷ 9 ∷ 13 ∷ 17 ∷ [])
  sum2by2-nats = refl
\end{code}
\end{frame}


\begin{frame}{A slightly more interesting example}
\textbf{Question.} What does the stream processor below do?
\begin{code}
  mutual
    sums : SP ℕ ℕ
    sums = get λ n → sumN n 0

    sumN : ℕ → ℕ → SP ℕ ℕ
    sumN zero a = put a λ where .force → sums
    sumN (suc n) a = get λ k → sumN n (a + k)
\end{code}
Let's run it on $\fun{nats}$!
\end{frame}

\begin{frame}{Composing stream processors}
If we have a $\data{SP}~A~B$ and a $\data{SP}~B~C$, we can apply them in sequence to a $\data{Stream}~A$ to get a $\data{Stream}~C$.

\textbf{Exercise.} Do the same with a single processor:
\begin{code}
  compose : SP A B → SP B C → SP A C
\end{code}
\begin{code}[hide]
  compose (put y f)  (get g)    = compose (f .force) (g y)
  compose (get f)    p2         = get (λ x → compose (f x) p2)
  compose p1         (put z g)  = put z (λ where .force → compose p1 (g .force))
\end{code}
\begin{code}
  compose-correct :
    (p1 : SP A B) (p2 : SP B C) (s : Stream A) →
    run (compose p1 p2) s ~ run p2 (run p1 s)
\end{code}
\begin{code}[hide]
  compose-correct (get f) p2 s = compose-correct (f (s .head)) p2 (s .tail)
  compose-correct (put y f) (get g) s = compose-correct (f .force) (g y) s
  compose-correct (put y f) (put z g) s .head = refl
  compose-correct (put y f) (put z g) s .tail = compose-correct (put y f) (g .force) s
\end{code}
\end{frame}

\section{Formal languages}

\begin{frame}{Formal languages, coinductively}
We can describe a formal language $l$ (= a set of strings) over an alphabet $A$ with two pieces of data:
\begin{itemize}
  \item whether it  is \alert{nullable} \\ (= contains the empty string)
  \item for each $a ∈ A$, the \alert{derivative} $\delta_a(l) = \{ s~|~a\cdot s ∈ l \}$.
\end{itemize}
Note that this is a \emph{coinductive} description of formal languages.
\end{frame}

\begin{frame}{Formal languages as (infinite) tries}
We can visualize a language as an infinite \alert{trie}:
\begin{center}\includegraphics[width=.6\textwidth]{trie.png}\end{center}
\end{frame}

\begin{frame}{Coinductive formal languages in Agda}
\begin{code}[hide]
open import Relation.Binary.Definitions using (DecidableEquality)
open import Relation.Nullary.Decidable using (Dec; yes; no; does)
open import Function.Base using (id; const; _∘_)
\end{code}
\begin{code}
module FormalLanguages
  (A : Set) (_≟_ : DecidableEquality A) where

  record Lang : Set where
    coinductive
    field
      ν : Bool
      δ : A → Lang
  open Lang public
\end{code}
\end{frame}

\begin{frame}{Some simple languages}
\begin{code}
  ∅ : Lang
  ∅ .ν = false
  ∅ .δ = λ _ → ∅

  ε : Lang
  ε .ν = true
  ε .δ = λ _ → ∅

  char : A → Lang
  char a .ν = false
  char a .δ b = if does (a ≟ b) then ε else ∅
\end{code}
\end{frame}

\begin{frame}{Language membership and tabulation}
\begin{code}[hide]
  open Lists using (List; []; _∷_)
\end{code}
\begin{code}
  _∋_ : Lang → List A → Bool
  l ∋ [] = l .ν
  l ∋ (x ∷ xs) = l .δ x ∋ xs

  trie : (List A → Bool) → Lang
  trie f .ν = f []
  trie f .δ a = trie (f ∘ (a ∷_))
\end{code}
\end{frame}

\begin{frame}{Operations on languages}
\begin{code}
  complement : Lang → Lang
  complement l .ν    = not (l .ν)
  complement l .δ x  = complement (l .δ x)

  _∪_ : Lang → Lang → Lang
  (l₁ ∪ l₂) .ν    = l₁ .ν ∨ l₂ .ν
  (l₁ ∪ l₂) .δ x  = l₁ .δ x ∪ l₂ .δ x

  _∩_ : Lang → Lang → Lang
  (l₁ ∩ l₂) .ν    = l₁ .ν ∧ l₂ .ν
  (l₁ ∩ l₂) .δ x  = l₁ .δ x ∩ l₂ .δ x
\end{code}
\end{frame}

\begin{frame}{Language concatenation}
We run into a problem when defining concatenation of languages:
\begin{code}[hide]
  {-# NON_TERMINATING #-}
\end{code}
\begin{code}
  _·_ : Lang → Lang → Lang
  (l₁ · l₂) .ν    = l₁ .ν ∧ l₂ .ν
  (l₁ · l₂) .δ x  = (if l₁ .ν then l₂ else ∅) ∪ (l₁ .δ x · l₂)
\end{code}
{\color{red}\underline{Error: Termination checking failed for \_·\_.}}
{\color{red}\underline{Problematic calls: $l₁~.δ~x~·~l₂$}}

The guardedness is obscured by the call to $\fun{∪}$.
\end{frame}


\begin{frame}{Sized types to the rescue}
\begin{code}[hide]
module SizedFormalLanguages
  (A : Set) (_≟_ : DecidableEquality A) where

  open Lists using (List; []; _∷_)
  open import Size
  open InductiveIdentity
  variable i : Size
\end{code}
\begin{code}
  record Lang (i : Size) : Set where
    coinductive
    field
      ν  : Bool
      δ  : {j : Size< i} → A → Lang j
  open Lang public
\end{code}
\end{frame}

\begin{code}[hide]
  _∋_ : Lang ∞ → List A → Bool
  l ∋ []        = l .ν
  l ∋ (x ∷ xs)  = l .δ x ∋ xs

  trie : (List A → Bool) → Lang i
  trie f .ν    = f []
  trie f .δ a  = trie (f ∘ (a ∷_))

  ∅ : Lang i
  ∅ .ν    = false
  ∅ .δ _  = ∅

  ε : Lang i
  ε .ν    = true
  ε .δ _  = ∅

  char : A → Lang i
  char a .ν    = false
  char a .δ b  = if does (a ≟ b) then ε else ∅

  complement : Lang i → Lang i
  complement l .ν    = not (l .ν)
  complement l .δ x  =
    complement (l .δ x)

  _∩_ : Lang i → Lang i → Lang i
  (l₁ ∩ l₂) .ν    = l₁ .ν ∧ l₂ .ν
  (l₁ ∩ l₂) .δ x  =
    l₁ .δ x ∩ l₂ .δ x
\end{code}

\begin{frame}{Language concatenation with sizes}
We can define union to be size-preserving:
\begin{code}
  _∪_ : Lang i → Lang i → Lang i
  (l₁ ∪ l₂) .ν    = l₁ .ν ∨ l₂ .ν
  (l₁ ∪ l₂) .δ x  = l₁ .δ x ∪ l₂ .δ x
\end{code}
This allows the definition of concatenation to pass:
\begin{code}
  _·_ : Lang i → Lang i → Lang i
  (l₁ · l₂) .ν    = l₁ .ν ∧ l₂ .ν
  (l₁ · l₂) .δ x  = (if l₁ .ν then l₂ else ∅) ∪ (l₁ .δ x · l₂)
\end{code}
\end{frame}

\begin{frame}{Definition of Kleene star}
\begin{code}
  _* : Lang i → Lang i
  (l *) .ν    = true
  (l *) .δ x  = l .δ x · (l *)
\end{code}
\end{frame}


\begin{frame}{Arden's rule}
\alert{Arden's rule} states: for a non-nullable language $k$, if $l = (k~\fun{·}~l)~\fun{∪}~m$, then $l = (k~\fun{*})~\fun{·}~m$

\textbf{Question.} How do we state this rule in Agda?
\pause
\textbf{Answer.} Using bisimulation!\footnote{or path equality in cubical Agda.}
\end{frame}

\begin{frame}{Bisimulation of languages}
\begin{code}[hide]
  infix 6 _~⟨_⟩~_
\end{code}
\begin{code}
  record _~⟨_⟩~_
    (l₁ : Lang ∞) (i : Size) (l₂ : Lang ∞) : Set where
    coinductive
    field
      ν  : l₁ .ν ≡ l₂ .ν
      δ  : {j : Size< i} (x : A) → l₁ .δ x ~⟨ j ⟩~ l₂ .δ x
  open _~⟨_⟩~_
\end{code}
\end{frame}

\begin{frame}{Arden's rule in Agda}
Now we can state Arden's rule:
\begin{code}
  arden : (k l m : Lang ∞) →
    l ~⟨ ∞ ⟩~ (k · l) ∪ m → l ~⟨ ∞ ⟩~ (k *) · m
\end{code}
\begin{code}[hide]
  arden = {!   !}
\end{code}
For the full proof, see \emph{Equational Reasoning about Formal Languages in
Coalgebraic Style} by Andreas Abel (2016).
\end{frame}


\begin{comment}

\begin{frame}{Exercises (part 1)}
\begin{itemize}
\item Prove reflexivity, symmetry, and transitivity of bisimulation.
\item Prove associativity, commutativity, and idempotence of \fun{\_∪\_}
\item Prove that \fun{∅} is neutral for \fun{\_∪\_}
\item Prove that \fun{\_∪\_} and \fun{\_·\_} preserve bisimulation
\item Prove that \fun{\_·\_} distributes over \fun{\_∪\_}
\item Prove that \fun{\_·\_} is associative
\end{itemize}
\end{frame}


\begin{frame}{Exercises (part 2)}
\begin{itemize}
\item Prove that \fun{∅} is absorbing for \fun{\_·\_}
\item Prove that \fun{ε} is neutral for \fun{\_·\_}
\item Prove that $\fun{∅}~\fun{*} = \fun{ε}$
\item Prove that $(l~\fun{*})~\fun{·}~(l~\fun{*}) = l~\fun{*}$
\item Prove that $(l~\fun{*})~\fun{*} = l~\fun{*}$
\item Prove that $l~\fun{*} = \fun{ε}~\fun{∪}~(l~\fun{·}~l~\fun{*})$
\item Prove
\end{itemize}
\end{frame}

\end{comment}

\begin{comment}

\section{Wander types}

\begin{frame}
\begin{code}
data ZeroOne : Set
record ZeroOne' : Set

data ZeroOne where
  Zero : ZeroOne' → ZeroOne
  One  : ZeroOne  → ZeroOne

record ZeroOne' where
  coinductive
  field force : ZeroOne
open ZeroOne' public

allZeros : ZeroOne
allZeros = Zero λ where .force → allZeros

alternate01 : ZeroOne
alternate01 = Zero (λ where .force → One alternate01)

-- allOnes : ZeroOne
-- allOnes = One allOnes
--               ^^^^^^^
-- Termination checking failed for the following functions:
--   allOnes
-- Problematic calls:
--   allOnes
\end{code}
\end{frame}
\end{comment}

\begin{comment}
\section{Coinductive graphs}

\begin{frame}

\end{frame}

\end{comment}

\section*{Conclusion}

\begin{frame}{References} \footnotesize
\begin{itemize}
\item Abel \& Pientka (JFP 2016): \emph{Well-founded recursion with copatterns and sized types.}
\item Ghani, Hancock \& Pattinson (LMCS 2009): \emph{Representations of stream processors using nested fixed points.}
\item Capretta (LMCS 2005): \emph{General Recursion Via Coinductive Types.}
\item Altenkirch, Danielsson \& Kraus (FoSSaCS 2017): \emph{Partiality, Revisited: The Partiality Monad as a Quotient Inductive-Inductive Type.}
\item Abel (2016): \emph{Equational Reasoning about Formal Languages in Coalgebraic Style.}
\item Capretta (): \emph{Wander types: A formalization of coinduction-recursion}
\item Kidney \& Wu (POPL 2025): \emph{Formalising Graph Algorithms with Coinduction}
\end{itemize}
\end{frame}


\end{document}






%%% Local Variables:
%%% mode: latex
%%% TeX-master: t
%%% LaTeX-command: "latex -shell-escape"
%%% TeX-command-extra-options: "-shell-escape"
%%% End:
