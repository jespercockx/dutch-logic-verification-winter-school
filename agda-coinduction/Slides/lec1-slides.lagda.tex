\documentclass[english,14pt,dvipsnames,table,fleqn]{beamer}

\usepackage{amsmath,amsfonts,amssymb}
\usepackage{appendixnumberbeamer}
\usepackage{array}
\usepackage{babel}
\usepackage[scaled=0.81]{beramono}
\usepackage{booktabs}
\usepackage[T1]{fontenc}
\usepackage{graphicx}
\usepackage[utf8]{inputenc}
\usepackage{lmodern}
\usepackage{mathtools}
\usepackage{minibox}
\usepackage{minted}  % Syntax highlighting using pygmentize
\usepackage{multicol}
\usepackage{pbox}
\usepackage{tabularx}
\usepackage{times}
\usepackage[normalem]{ulem}
\usepackage{wasysym}
\usepackage{xparse}
\usepackage{xspace}
\usepackage[utf8]{inputenc}
\usepackage{agda}
\usepackage{etoolbox}

\usepackage{minted}
\newminted[Haskell]{haskell}{mathescape,escapeinside=~~}
\newmintinline[Hs]{haskell}{}
\definecolor{markinlineColor}{RGB}{204,229,255}
\newmintinline[Hc]{haskell}{bgcolor=markinlineColor}


\newminted[Java]{java}{mathescape,escapeinside=~~}

\newminted[JavaScript]{javascript}{mathescape,escapeinside=~~}


\usepackage{color,colortbl,xcolor}
\usepackage{comment}

\usepackage{tikz}
\usetikzlibrary{arrows,arrows.meta}
\usetikzlibrary{calc}
\usetikzlibrary{fit}
\usetikzlibrary{overlay-beamer-styles}
\usetikzlibrary{positioning}
\usetikzlibrary{shapes,shapes.symbols}
\usetikzlibrary{tikzmark}
\usetikzlibrary{trees}

\definecolor{mc1}{RGB}{27,158,119}
\definecolor{mc2}{RGB}{217,95,2}
\definecolor{mc3}{RGB}{117,112,179}

\usepackage{pgfplots}
\usepgfplotslibrary{dateplot}

\usepackage[absolute,overlay]{textpos}  % place images with absolute position
  \setlength{\TPHorizModule}{1mm}
  \setlength{\TPVertModule}{1mm}

% Use metropolis beamer theme
\usetheme{metropolis}
\usepackage{FiraSans}  % Really nice font for metro theme
\setbeamercolor{progress bar}{fg=darkgray}

\setbeamertemplate{itemize item}{$\bullet$}
\setbeamertemplate{navigation symbols}{}
\setbeamertemplate{footline}[frame number]

% Define block styles
\tikzstyle{decision} = [diamond, draw, fill=blue!20, text width=4.5em,
  text badly centered, node distance=3cm, inner sep=0pt]
\tikzstyle{block} = [rectangle, draw, fill=mDarkTeal, font=\footnotesize,
  text=white, text width=4em, text centered, rounded corners, minimum height=3em]
\tikzstyle{line} = [draw, -latex']
\tikzstyle{cloud} = [draw, ellipse,fill=mDarkTeal!20, node distance=2.5cm,
  font=\footnotesize, minimum height=2em]

% Stuff for the title page
\title{\lecturetitle}
\author{Jesper Cockx}
\institute{Technical University Delft}
%\titlegraphic{\includegraphics[height=2cm]{TUDelft.png}}


% `to' (destination) node in code
\def\soToNode#1{\node (#1) at ($(pic cs:#1) + (-3pt,0)$) {}}
% `from' (comment) node
\DeclareDocumentCommand{\soFromNode}{ m m m }{\node (_#1) [#3 #1] {#2}}
% arrow connecting `from' and `to'
\DeclareDocumentCommand{\soArrow}{ m m }{\draw [#2] (_#1) -> (#1)}
% #1             #2                   #3                 #4                #5
% tikzmark name, text in description, where description, arrow attributes, arguments to tikzpicture
\DeclareDocumentCommand{\soComment}{ m m O{below=4mm of} O{black,very thick} O{} }{%
\begin{tikzpicture}[remember picture, ->, >=stealth, overlay, red, very thick, align=left,#5]
  \soToNode{#1};
  \soFromNode{#1}{#2}{#3};
  \soArrow{#1}{#4};
\end{tikzpicture}
}

\tikzset{var/.style={draw,rectangle,blue,very thick,minimum size=30}}
\tikzset{obj/.style={draw,ellipse,minimum width=30,thick}}

\tikzset{cls/.style={draw,ellipse,minimum width=70,very thick}}
\tikzset{int/.style={draw,ellipse,minimum width=70,very thick, dotted}}

% http://tex.stackexchange.com/questions/100313/a-problem-with-beamer-tikz-and-fonts
\let\next\relax  % See URL above for why this is needed with TikZ + Beamer


\setbeamertemplate{section in toc}[square]

\makeatletter
\patchcmd{\beamer@sectionintoc}{\vskip1.5em}{\vskip0.5em}{}{}
\makeatother

\AtBeginSection[]
{
  \begin{frame}<beamer>[plain,noframenumbering]{Outline}
    \tableofcontents[currentsection,hideallsubsections]
  \end{frame}
}

\makeatletter
\newlength\beamerleftmargin
\setlength\beamerleftmargin{\Gm@lmargin}
\makeatother

\renewcommand{\arraycolsep}{3pt}

\newdimen\mathindent
\mathindent=2ex


\usepackage{newunicodechar}
\newunicodechar{→}{\ensuremath{\rightarrow}}
\newunicodechar{←}{\ensuremath{\leftarrow}}
\newunicodechar{×}{\ensuremath{\times}}
\newunicodechar{λ}{\ensuremath{\lambda}}
\newunicodechar{∀}{\ensuremath{\forall}}
\newunicodechar{Π}{\ensuremath{\Pi}}
\newunicodechar{Σ}{\ensuremath{\Sigma}}
\newunicodechar{≡}{\ensuremath{\equiv}}
\newunicodechar{≅}{\ensuremath{\cong}}
\newunicodechar{≐}{\ensuremath{\doteq}}
\newunicodechar{∈}{\ensuremath{\in}}
\newunicodechar{∧}{\ensuremath{\land}}
\newunicodechar{∨}{\ensuremath{\lor}}
\newunicodechar{⊤}{\ensuremath{\top}}
\newunicodechar{⊥}{\ensuremath{\bot}}
\newunicodechar{⊔}{\ensuremath{\sqcup}}
\newunicodechar{∷}{\ensuremath{{::}}}
\newunicodechar{ℓ}{\ensuremath{\ell}}
\newunicodechar{₀}{\ensuremath{{_0}}}
\newunicodechar{₁}{\ensuremath{{_1}}}
\newunicodechar{₂}{\ensuremath{{_2}}}
\newunicodechar{₃}{\ensuremath{{_3}}}
\newunicodechar{₄}{\ensuremath{{_4}}}
\newunicodechar{₅}{\ensuremath{{_5}}}
\newunicodechar{₆}{\ensuremath{{_6}}}
\newunicodechar{₇}{\ensuremath{{_7}}}
\newunicodechar{₈}{\ensuremath{{_8}}}
\newunicodechar{₉}{\ensuremath{{_9}}}
\newunicodechar{⟨}{\ensuremath{{\langle}}}
\newunicodechar{⟩}{\ensuremath{{\rangle}}}
\newunicodechar{̧}{\c}
\newunicodechar{∘}{\ensuremath{\circ}}
\newunicodechar{≤}{\ensuremath{\leq}}
\newunicodechar{⋯}{\ensuremath{\cdots}}
\newunicodechar{ℕ}{\ensuremath{\mathbb{N}}}
\newunicodechar{∎}{\ensuremath{\blacksquare}}
\newunicodechar{∞}{\ensuremath{\infty}}
\newunicodechar{⊎}{\ensuremath{\uplus}}
\newunicodechar{¬}{\ensuremath{\lnot}}
\newunicodechar{⇓}{\ensuremath{\Downarrow}}
\newunicodechar{≟}{\ensuremath{\stackrel{?}{=}}}
\newunicodechar{ν}{\ensuremath{\nu}}
\newunicodechar{δ}{\ensuremath{\delta}}
\newunicodechar{∅}{\ensuremath{\emptyset}}
\newunicodechar{ε}{\ensuremath{\varepsilon}}
\newunicodechar{∋}{\ensuremath{\ni}}
\newunicodechar{∩}{\ensuremath{\cap}}
\newunicodechar{∪}{\ensuremath{\cup}}
\newunicodechar{·}{\ensuremath{\cdot}}

\newcommand\var[1]{\mathit{#1}}
\newcommand\prim[1]{{\AgdaPrimitive{#1}}}
\newcommand\ty[1]{{{\prim{Set}}_{#1}}}
\newcommand\fun[1]{{\AgdaFunction{#1}}}
\newcommand\data[1]{{\AgdaFunction{#1}}}
\newcommand\con[1]{{\AgdaInductiveConstructor{#1}}}
\newcommand\field[1]{{\AgdaField{#1}}}
\newcommand\keyw[1]{{\AgdaKeyword{#1}}}
\newcommand\lit[1]{{\AgdaNumber{#1}}}
\newcommand\level{\fun{Level}}
\newcommand\ite[3]{\fun{if}\  #1\  \fun{then}\  #2\  \fun{else}\  #3}


\newcommand\Nat{\data{ℕ}}
\newcommand\zero{\con{zero}}
\newcommand\suc{\con{suc}}
\newcommand\Bool{\data{Bool}}
\newcommand\true{\con{true}}
\newcommand\false{\con{false}}
\newcommand\List{\data{List}}
\renewcommand\Vec{\data{Vec}}
\newcommand\nil{\con{[]}}
\newcommand\cons{\con{::}}
\newcommand\Fin{\data{Fin}}
\renewcommand\prod{\data{×}}
\newcommand\sigmatype{\data{Σ}}
\newcommand\toptype{\data{⊤}}
\newcommand\bottomtype{\data{⊥}}
\newcommand\Id{\data{≡}}
\newcommand\refl{\con{refl}}

\newcommand\prf[1]{\colorbox{gray!30}{#1}}
\newcommand{\ru}[2]{\dfrac{\begin{array}{@{}c@{}}#1\end{array}}{\begin{array}{@{}c@{}}#2\end{array}}}



%%% Local Variables:
%%% mode: latex -shell-escape
%%% TeX-master: t
%%% End:


\def\lecturetitle{Coinductive Programming and Proving in Agda}
\newcommand\courseyear{2026}
\subtitle{Lecture 1: Coinductive programming in Agda}
\date{21 January 2026}

\begin{document}

\maketitle

\begin{frame}[plain,noframenumbering]{Lecture plan}
  \tableofcontents[hideallsubsections]
\end{frame}

\section{Introduction}


\begin{frame}{Why dependent types?}
Dependent types embed formal verification \alert{directly inside your type system}.

\textbf{Advantages of dependent types.}
\begin{itemize}
  \item Single syntax for programming and proving
  \item Editor support for interactive development
  \item Invariants can be embedded inside programs (= \emph{intrinsic} verification)
\end{itemize}

Verifying a program should not be more difficult than writing it in the first place!
\end{frame}

\begin{frame}{The Agda language}
\begin{center}\includegraphics[width=.5\textwidth]{Agda.jpg}\end{center}

Agda is a \alert{purely functional} programming language similar to Haskell.

Unlike Haskell, it has full support for \alert{dependent types}.

It also supports \alert{interactive programming} with help from the type checker.
\end{frame}

\begin{frame}{Installing Agda}
\begin{enumerate}
\item \textbf{Agda binary.} Download the binary\footnote{\href{https://github.com/agda/agda/releases/tag/v2.8.0}{github.com/agda/agda/releases/tag/v2.8.0}}, then run \texttt{agda -{}-setup}.

\item \textbf{Editor plugin.} Install the VS Code plugin or run \texttt{agda -{}-emacs-mode setup}.

\item \textbf{Standard library.} Download and unpack\footnote{\href{https://github.com/agda/agda-stdlib/archive/v2.3.tar.gz}{github.com/agda/agda-stdlib/archive/v2.3.tar.gz}}, then add it to \texttt{libraries} and \texttt{defaults}.
\end{enumerate}
Detailed instructions: \href{https://agda.readthedocs.io/en/v2.8.0/getting-started/installation.html}{agda.readthedocs.io/en/v2.8.0/getting-started}
\end{frame}


\begin{code}[hide]
{-# OPTIONS --allow-unsolved-metas --guardedness --sized-types --cubical -WnoUnsupportedIndexedMatch #-}
module _ where

data Either (A B : Set) : Set where
  left   : A → Either A B
  right  : B → Either A B

\end{code}


\begin{frame}{Basic syntax}
\textbf{Names} can be any non-reserved sequence of unicode\footnote{Supported editors will replace LaTeX-like syntax (e.g.~\texttt{\textbackslash{}to}) with unicode.} characters, except \texttt{.;\{\}()\@"}.

A \textbf{type declaration} is written as $\var{b}~:~\var{A}$.

\textbf{Function types} are written $A → B$ or $(x : A) → B$. $λ\ x → u$ is an (anonymous) function and $f\ x$ is function application.

An \textbf{infix operator} $\fun{\_+\_}$ is used as $x\ +\ y$. \\ $\fun{x+y}$ is a valid name, so use enough spaces.

\end{frame}

\begin{frame}{Loading an Agda file}
  You can \alert{load} an Agda file by pressing \texttt{Ctrl+c} followed by \texttt{Ctrl+l}.

  Once the file is loaded (and there are no errors), other commands become available:
  \begin{description}
    \item[\texttt{Ctrl+c Ctrl+d}] \hfill Infer type of an expression.
    \item[\texttt{Ctrl+c Ctrl+n}] \hfill Evaluate an expression.
  \end{description}
\end{frame}

\begin{frame}{Holes in programs}
  A \alert{hole} is an incomplete part of a program.
  You can create one by writing \texttt{?} or \texttt{\{!!\}} and loading the file.

  New commands for holes:
  \begin{description}
    \item[\texttt{Ctrl+c Ctrl+,}] \hfill Get hole information
    \item[\texttt{Ctrl+c Ctrl+c}] \hfill Case split on a variable
    \item[\texttt{Ctrl+c Ctrl+space}] \hfill Fill in the hole
  \end{description}
\end{frame}

\section{Data types and (co)pattern matching}

\begin{frame}{Declaring new datatypes}
\begin{code}
data Bool : Set where
  true   : Bool
  false  : Bool

data ℕ : Set where
  zero  : ℕ
  suc   : ℕ → ℕ
{-# BUILTIN NATURAL ℕ #-}
\end{code}
$\ty{}$ is the type of (small) types (see later).
\end{frame}

\begin{frame}{Defining functions by pattern matching}
\begin{code}
not : Bool → Bool
not true   = false
not false  = true

_+_ : ℕ → ℕ → ℕ
zero     + y  = y
(suc x)  + y  = suc (x + y)
\end{code}
\textbf{Exercise.} Define \fun{isEven}, \fun{\_*\_}, and \fun{\_≤\_} on \Nat.
\end{frame}


\begin{frame}{Total functional programming}
  Agda is a \alert{total} language: evaluating a function call always returns a result in
finite time:
  \begin{itemize}
    \item The \textbf{coverage checker} ensures completeness of pattern matches.
    \item The \textbf{termination checker} ensures termination of recursive definitions.
  \end{itemize}

  Reasons to care about totality:
  \begin{itemize}
    \item It prevents crashes and infinite loops.
    \item It is needed for \textbf{decidable type checking}.
    \item It is needed for \textbf{logical soundness}.
  \end{itemize}
\end{frame}


\begin{code}[hide]
isEven : ℕ → Bool
isEven zero           = true
isEven (suc zero)     = false
isEven (suc (suc x))  = isEven x

half : ℕ → ℕ
half zero = zero
half (suc zero) = zero
half (suc (suc n)) = suc (half n)

_*_ : ℕ → ℕ → ℕ
zero    * y = zero
(suc x) * y = (x * y) + y

_≤_ : ℕ → ℕ → Bool
zero   ≤ y      = true
suc x  ≤ zero   = false
suc x  ≤ suc y  = x ≤ y

infixl 10 _+_
infixl 20 _*_

myNumber = 1 + 2 * 3 + 4
-- Parsed as ((1 + (2 * 3)) + 4
-- With infixr, it would be parsed as
-- 1 + ((2 * 3) + 4) instead.
\end{code}

\begin{frame}{Record types and projections}
\begin{code}
record Rect : Set where
  constructor rect -- optional
  field
    height : ℕ
    width  : ℕ

square : ℕ → Rect
square x = record { height = x ; width = x }
    -- or: rect x x

area : Rect → ℕ
area r = Rect.height r + Rect.width r
\end{code}
\end{frame}

\begin{frame}{More record syntax}
You can \alert{open} the record to bring projections into scope:
\begin{code}
open Rect

perimeter : Rect → ℕ
perimeter r = 2 * height r + 2 * width r
\end{code}
You can also use projections as \alert{postfix}:
\begin{code}
rotate : Rect → Rect
rotate r = rect (r .width) (r .height)
\end{code}
\end{frame}

\begin{frame}{Copattern matching}
Where data types are defined by how we \alert{construct} an element,
record types are defined by what we can \alert{observe} of an element.

This duality is exploited by Agda's \alert{copattern} syntax:
\begin{code}[hide]
stretch : Rect → Rect
height  (stretch r)   = 2 * height r
width   (stretch r)   = half (width r)
\end{code}

\begin{code}
squeeze : Rect → Rect
squeeze r  .height  = half (r .height)
squeeze r  .width   = 2 * r .width
\end{code}
We define $\fun{squeeze}~r$ by its projections.
\end{frame}




\section{Universes and polymorphism}

\begin{frame}{The type \texttt{Set}}
In Agda, types such as $\Nat$ and $\Bool \to \Bool$ are themselves
expressions of type $\ty{}$.

We can pass around and return values of type $\ty{}$ just like values
of any other type.

\textbf{Example.}
\begin{code}[hide]
module ExplicitId where
\end{code}
\begin{code}
  id : (A : Set) → A → A
  id A x = x
\end{code}
A type like $\ty{}$ whose elements are themselves types is called a \alert{universe}.
\end{frame}

\begin{frame}{Side note: the \texttt{Set} hierarchy}
$\ty{}$ itself does not have type $\ty{}$: assuming so leads to inconsistency.\footnote{See Girard's paradox and Hurken's paradox.}

Instead, $\ty{} = \ty{0}$ has type $\ty{1}$, which has type $\ty{2}$, which has type\ldots

In fact, you can write \alert{universe-polymorphic} definitions by quantifying over $l : \data{Level}$ and working with universe $\ty{}~l$.
\end{frame}

\begin{frame}{Polymorphic functions in Agda}
We can use $\ty{}$ to define polymorphic functions and data types:
\begin{code}
data List (A : Set) : Set where
  []    : List A
  _∷_  : A → List A → List A

length : {A : Set} → List A → ℕ
length [] = 0
length (_ ∷ xs) = suc (length xs)
\end{code}
The curly braces mark $A$ as \alert{implicit}.

%\textbf{Note.} This is a first example of a \alert{dependent function}:
%the type of $\fun{id}$ \emph{depends on} the value of its first argument.
\textbf{Exercise.} Implement \fun{map} and \fun{\_++\_}.
\end{frame}

\begin{code}[hide]
infixr 5 _∷_


map : {A B : Set} → (A → B) → List A → List B
map f []         = []
map f (x ∷ xs)  = f x ∷ map f xs

_++_ : {A : Set} → List A → List A → List A
[]        ++  ys  = ys
(x ∷ xs)  ++  ys  = x ∷ (xs ++ ys)

infixr 15 _++_
\end{code}


\begin{frame}{Variable generalization}
We can mark declare a variable to be generalized automatically:
\begin{code}
variable A B C : Set

id : A → A
id x = x
\end{code}
This is equivalent to writing $\{A : \ty{}\} → \ldots$
\end{frame}


\begin{frame}{Polymorphic record types}
\begin{code}
record _×_ (A B : Set) : Set where
  constructor _,_
  field
    proj₁ : A
    proj₂ : B
open _×_ public

swap : A × B → B × A
swap (x , y) = y , x
\end{code}
\end{frame}

\begin{frame}{If/then/else as a function}
We can define if/then/else in Agda as follows:
\begin{code}
if_then_else_ : Bool → A → A → A
if true   then x else y  = x
if false  then x else y  = y
\end{code}
This is an example of a \alert{mixfix operator}.

\textbf{Example usage.}
\begin{code}
test : ℕ → ℕ
test x = if (x ≤ 9000) then 0 else 42
\end{code}
\end{frame}

\begin{frame}{The empty type and absurd patterns}

\begin{code}
data ⊥ : Set where

absurd : ⊥ → A
absurd ()
\end{code}
The \alert{absurd pattern} \texttt{()} indicates that there are no possible constructors.
\end{frame}

\begin{frame}{The disjoint sum type}
The \data{Either} type from Haskell is called \data{\_⊎\_}:
\begin{code}
data _⊎_ (A B : Set) : Set where
  inj₁  : A → A ⊎ B
  inj₂  : B → A ⊎ B

[_,_] : (A → C) → (B → C) → A ⊎ B → C
[ f , g ] (inj₁ x)  = f x
[ f , g ] (inj₂ y)  = g y
\end{code}
\end{frame}

\section{Dependent types}

\begin{frame}{Dependent types}
A \alert{dependent type} is a family of types, depending on a term of a \alert{base type}: \small
\begin{code}
data Flavour : Set where
  cheesy chocolatey : Flavour

data Food : Flavour → Set where
  pizza  : Food cheesy
  cake   : Food chocolatey
  bread  : {f : Flavour} → Food f

amountOfCheese : Food cheesy → ℕ
amountOfCheese pizza  = 100
amountOfCheese bread  = 20
\end{code}
\normalsize
Agda knows that $\con{cake}$ is not a valid input!
\end{frame}


\begin{frame}{Vectors: lists that know their length}
$\data{Vec}\ A\ n$ is the type of \alert{vectors} of length $n$:
\begin{code}[hide]
variable
  m n : ℕ
\end{code}
\begin{code}
data Vec (A : Set) : ℕ → Set where
  []    : Vec A 0
  _∷_   : A → Vec A n → Vec A (suc n)
\end{code}
\begin{code}
myVec : Vec ℕ 4
myVec = 1 ∷ 2 ∷ 3 ∷ 4 ∷ []
\end{code}
Types will be normalized during type checking:
\begin{code}
myVec' : Vec ℕ (2 + 2)
myVec' = myVec
\end{code}
\end{frame}

\begin{frame}{Side note: Parameters vs.~indices}
The argument $(A : Set)$ in the definition of $\data{Vec}$ is a \alert{parameter},
and has to be \emph{the same in the type of each constructor}.

The argument of type $\Nat$ in the definition of $\data{Vec}$ is an \alert{index},
and must be \emph{determined individually for each constructor}.
\end{frame}



\begin{frame}{Quiz question}
  \textbf{Question.} How many elements are there in the type $\data{Vec}\ \data{Bool}\ \lit{3}$? \pause

  \textbf{Answer.} 8 elements: \footnotesize
  \begin{itemize}
    \item $\true~\con{∷}~\true~\con{∷}~\true~\con{∷}~\con{[]}$
    \item~$\true~\con{∷}~\true~\con{∷}~\false~\con{∷}~\con{[]}$
    \item~$\true~\con{∷}~\false~\con{∷}~\true~\con{∷}~\con{[]}$
    \item~$\true~\con{∷}~\false~\con{∷}~\false~\con{∷}~\con{[]}$
    \item~$\false~\con{∷}~\true~\con{∷}~\true~\con{∷}~\con{[]}$
    \item~$\false~\con{∷}~\true~\con{∷}~\false~\con{∷}~\con{[]}$
    \item~$\false~\con{∷}~\false~\con{∷}~\true~\con{∷}~\con{[]}$
    \item~$\false~\con{∷}~\false~\con{∷}~\false~\con{∷}~\con{[]}$
  \end{itemize}
\end{frame}


\begin{frame}{Dependent function types}

A \alert{dependent function type} is a type of the form $(x : A) \to B\ x$
where the \emph{type} of the output depends on the \emph{value} of the input.

\textbf{Example.}
\begin{code}
replicate : (n : ℕ) → A → Vec A n
replicate zero     x  = []
replicate (suc n)  x  = x ∷ replicate n x
\end{code}
E.g.~$\fun{replicate}\ \lit{3}\ \lit{0}$ has type $\data{Vec}\ \Nat\ \lit{3}$ and evaluates to $\lit{0}\ \cons\ \lit{0}\ \cons\ \lit{0}\ \cons\ \nil$.
\end{frame}



\begin{frame}{Dependent pattern matching}
We can pattern match on $\data{Vec}$ just like on $\data{List}$:
\begin{code}
mapV : (A → B) → Vec A n → Vec B n
mapV f []         = []
mapV f (x ∷ xs)   = f x ∷ mapV f xs

head : Vec A (suc n) → A
head (x ∷ xs) = x

tail : Vec A (suc n) → Vec A n
tail (x ∷ xs) = xs
\end{code}
In $\fun{head}$ and $\fun{tail}$, the cases for $\nil$ are impossible!
\end{frame}


\begin{frame}[t]{A safe lookup}

To define a total lookup on vectors, we need the type $\data{Fin}\ n$:

\begin{code}
data Fin : ℕ → Set where
  zero  : Fin (suc n)
  suc   : Fin n → Fin (suc n)

lookupVec : Vec A n → Fin n → A
lookupVec (x ∷ xs)  zero     =  x
lookupVec (x ∷ xs)  (suc i)  =  lookupVec xs i
\end{code}

Again, there is no case for the empty vector!
\end{frame}

\begin{frame}{The family of \texttt{Fin} types}
\begin{tikzpicture}[node distance=0.5cm and 0.3cm, auto]
  \node (fz) {$\Fin\ \lit{0}$};
  \node (fsz) [right=of fz] {$\Fin\ \lit{1}$};
  \node (fssz) [right=of fsz] {$\Fin\ \lit{2}$};
  \node (fsssz) [right=of fssz] {$\Fin\ \lit{3}$};
  \node (fssssz) [right=of fsssz] {$\Fin\ \lit{4}$};
  \node (fsssssz) [right=of fssssz] {$\cdots$};
  \node (z1) [below=of fsz] {$\zero$};
  \node (z2) [below=of fssz] {$\zero$};
  \node (z3) [below=of fsssz] {$\zero$};
  \node (z4) [below=of fssssz] {$\zero$};
  \node (z5) [below=of fsssssz] {$\cdots$};
  \node (sz2) [below=of z2] {$\suc$};
  \node (sz3) [below=of z3] {$\suc$};
  \node (sz4) [below=of z4] {$\suc$};
  \node (sz5) [below=of z5] {$\cdots$};
  \node (ssz3) [below=of sz3] {$\suc$};
  \node (ssz4) [below=of sz4] {$\suc$};
  \node (ssz5) [below=of sz5] {$\cdots$};
  \node (sssz4) [below=of ssz4] {$\suc$};
  \node (sssz5) [below=of ssz5] {$\cdots$};
  \draw[->] (sz2) to (z1);
  \draw[->] (sz3) to (z2);
  \draw[->] (sz4) to (z3);
  \draw[->] (ssz3) to (sz2);
  \draw[->] (ssz4) to (sz3);
  \draw[->] (sssz4) to (ssz3);
\end{tikzpicture}
\end{frame}

\begin{frame}{Some more vector functions}
\textbf{Exercise.} Implement the following functions:
\begin{code}
zipVec : Vec A n → Vec B n → Vec (A × B) n
updateVecAt : Fin n → A → Vec A n → Vec A n
\end{code}
\begin{code}[hide]
zipVec = {!   !}
updateVecAt = {!   !}
\end{code}
\end{frame}

\section{Coinductive record types}

\begin{frame}{Streams as a coinductive record}
A stream consists of a \alert{head} and a tail (= another stream):
\begin{code}[hide]
module Guardedness where
\end{code}
\begin{code}
  record Stream (A : Set) : Set where
    coinductive
    field
      headS  : A
      tailS  : Stream A
  open Stream public
\end{code}
The \alert{coinductive} keyword indicates that we want the largest such type.
\end{frame}

\begin{frame}{Functions on streams}
We can use the projections to define functions on streams:
\begin{code}
  firstTwo : Stream A → A × A
  firstTwo s = s .headS , s .tailS .headS

  dropS : ℕ → Stream A → Stream A
  dropS zero     s  = s
  dropS (suc n)  s  = dropS n (s .tailS)
\end{code}
\end{frame}

\begin{code}[hide]
  takeS : ℕ → Stream A → List A
  takeS zero s = []
  takeS (suc n) s = s .headS ∷ takeS n (s .tailS)

  _∷S_ : A → Stream A → Stream A
  (x ∷S xs) .headS = x
  (x ∷S xs) .tailS = xs

  _++S_ : List A → Stream A → Stream A
  [] ++S ys = ys
  (x ∷ xs) ++S ys = x ∷S (xs ++S ys)
\end{code}

\begin{frame}{Defining a new stream}
\begin{code}[hide]
  module bad where
    {-# NON_TERMINATING #-}
\end{code}
Defining a new stream with a record constructor fails the termination check:
\begin{code}
    zeroes : Stream ℕ
    zeroes = record { headS = 0 ; tailS = zeroes }
\end{code}
{\color{red}\underline{Termination checking failed for zeroes}}

Allowing this would violate strong normalization of Agda!
\end{frame}

\begin{frame}{Defining a new stream}
To define a new stream, we have to use copatterns instead:
\begin{code}
  zeroes : Stream ℕ
  zeroes .headS  = 0
  zeroes .tailS  = zeroes
\end{code}
\fun{zeroes} only reduces when projections are applied to it,
thus preserving strong normalization.
\end{frame}

\begin{frame}{The guardedness criterion}
Coinductive values should be \alert{productive}:
applying any (finite) number of projections to them should terminate.

This is enforced by the \alert{guardedness criterion}:
\footnote{Enabled by \texttt{\{-\# OPTIONS -{}-guardedness \#-\}}}
every (co)recursive call needs to appear
\begin{enumerate}
  \item in a clause with a copattern, \emph{and}
  \item either at the top level or as the argument to one or more constructors
\end{enumerate}
\end{frame}

\begin{frame}{Limitations of guardedness}
\begin{code}
  mapS : (A → B) → Stream A → Stream B
  mapS f xs .headS  = f (xs .headS)
  mapS f xs .tailS  = mapS f (xs .tailS)
\end{code}
\begin{code}[hide]
  module failingnat where
    {-# NON_TERMINATING #-}
\end{code}
\begin{code}
    nats : Stream ℕ
    nats .headS = 0
    nats .tailS = mapS suc nats
\end{code}
{\color{red}\underline{Termination checking failed for nats}}

\textbf{Problem.} The guardedness checker does not know anything about \fun{mapS}!
\end{frame}

\begin{frame}{Working around the limitations}
\begin{code}
  natsFrom : ℕ → Stream ℕ
  natsFrom n .headS = n
  natsFrom n .tailS = natsFrom (suc n)

  nats : Stream ℕ
  nats = natsFrom 0
\end{code}
Alternatively, we can use \alert{sized types} to provide Agda with more information about \fun{mapS} (see later).
\end{frame}

\begin{frame}{Exercises}
\begin{itemize}
\item Define $\fun{repeat} : A \to \data{Stream}~A$.
\item Define $\fun{lookup} : \data{Stream}~A \to \Nat \to A$.
\item Define $\fun{tabulate} : (\Nat \to A) \to \data{Stream}~A$.
\item Use $\fun{tabulate}$ to define $\fun{fibonacci} : \data{Stream}~\Nat$.
\item Define $\fun{transpose} : \data{Stream}~(\data{Stream}~A) \to \data{Stream}~(\data{Stream}~A)$.
\item \textbf{(Bonus)} Prove that $\fun{lookup}~(\fun{lookup}~(\fun{transpose}~\var{xss})~i)~j ≡ \fun{lookup}~(\fun{lookup}~\var{xss}~j)~i$.
\end{itemize}
\end{frame}

\section{Mixing induction and coinduction}

\begin{frame}{Colists: potentially infinite lists}
\begin{code}
  mutual
    data Colist (A : Set) : Set where
      []   : Colist A
      _∷_  : A → Colist' A → Colist A

    record Colist' (A : Set) : Set where
      coinductive
      field
        force : Colist A
  open Colist' public
\end{code}
\end{frame}

\begin{frame}{Converting a stream to a colist}
\begin{code}
  fromStream : Stream A → Colist A
  fromStream {A} xs = xs .headS ∷ rest
    where
      rest : Colist' A
      rest .force = fromStream (xs .tailS)
\end{code}
Alternatively, we use a \alert{copattern lambda}:
\begin{code}
  fromStream' : Stream A → Colist A
  fromStream' {A} xs = xs .headS ∷
    (λ where .force → fromStream' (xs .tailS))
\end{code}
\textbf{Exercise.} Define $\fun{fromList} : \data{List}~A → \data{Colist}~A$.
\begin{code}[hide]
  fromList : List A → Colist A
  fromList [] = []
  fromList (x ∷ xs) = x ∷ λ where .force → fromList xs
\end{code}
\end{frame}

\begin{frame}{Another example: co-natural numbers}
\begin{code}
  mutual
    data Coℕ : Set where
      zero  : Coℕ
      suc   : Coℕ' → Coℕ

    record Coℕ' : Set where
      coinductive
      field force : Coℕ
  open Coℕ' public
\end{code}
\textbf{Exercise.} Define $\fun{∞} : \data{Coℕ}$, $\fun{fromℕ} : ℕ → \data{Coℕ}$, and $\fun{colength} : \data{Colist}~A → \data{Coℕ}$.
\end{frame}


\section{Sized types}

\begin{frame}{Sized types}
\alert{Sized types}\footnote{Enabled by \texttt{\{-\# OPTIONS -{}-sized-types \#-\}}} are an alternative to the syntactic guardedness checker that annotates the size of expressions in their types.

The module $\texttt{Size}$ provides:
\begin{itemize}
  \item $\data{Size} : \ty{}$
  \item $\data{Size<} : \data{Size} → \ty{}$
  \item $\fun{∞} : \data{Size}$.
\end{itemize}
+ some operators not relevant for coinduction.
\end{frame}

\begin{frame}{Sized streams}
\begin{code}[hide]
module SizedTypes where
  open import Size
  variable i : Size
\end{code}
We can parametrize a stream by its size $i$\\
(= the number of elements we can observe):
\begin{code}
  record Stream (A : Set) (i : Size) : Set where
    coinductive
    field
      headS  : A
      tailS  : {j : Size< i} → Stream A j
  open Stream
\end{code}
The tail can be assigned any size $j$ that is strictly smaller than $i$.
\end{frame}

\begin{frame}{The infinite size}
When \emph{consuming} a stream we can use size $∞$:
\begin{code}
  takeS : ℕ → Stream A ∞ → List A
  takeS zero s     = []
  takeS (suc n) s  = s .headS ∷ takeS n (s .tailS)
\end{code}
\textbf{Exercise.} Define $\fun{dropS}$.
\end{frame}

\begin{frame}{Defining sized streams}
When \emph{defining} a new stream we should define it at an arbitrary size $i$:
\begin{code}
  zeroes : Stream ℕ i
  zeroes .headS  = 0
  zeroes .tailS  = zeroes
\end{code}
More explicitly:
\begin{code}
  zeroes' : {i : Size} → Stream ℕ i
  zeroes' {i} .headS      = 0
  zeroes' {i} .tailS {j}  = zeroes {j}
\end{code}
\end{frame}

\begin{frame}{A size-preserving map}
We can now give a more precise type to \fun{map}:
\begin{code}
  mapS : (A → B) → Stream A i → Stream B i
  mapS f s .headS  = f (s .headS)
  mapS f s .tailS  = mapS f (s .tailS)
\end{code}
This allows a direct definition of \fun{nats}:
\begin{code}
  nats : Stream ℕ i
  nats .headS  = 0
  nats .tailS  = mapS suc nats
\end{code}
\textbf{Exercise.} Define \fun{zipWithS} and use it to define \fun{fibonacci} without using \fun{tabulate}.
\end{frame}

\begin{frame}{Next time}
\begin{itemize}
  \item The Curry-Howard correspondence
  \item Equational reasoning
  \item Bisimulation as a coinductive relation
  \item Bisimulation as the cubical path type
\end{itemize}
\end{frame}





\end{document}






%%% Local Variables:
%%% mode: latex
%%% TeX-master: t
%%% LaTeX-command: "latex -shell-escape"
%%% TeX-command-extra-options: "-shell-escape"
%%% End:
