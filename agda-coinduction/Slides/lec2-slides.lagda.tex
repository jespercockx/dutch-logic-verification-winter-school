\documentclass[english,14pt,dvipsnames,table,fleqn]{beamer}

\usepackage{amsmath,amsfonts,amssymb}
\usepackage{appendixnumberbeamer}
\usepackage{array}
\usepackage{babel}
\usepackage[scaled=0.81]{beramono}
\usepackage{booktabs}
\usepackage[T1]{fontenc}
\usepackage{graphicx}
\usepackage[utf8]{inputenc}
\usepackage{lmodern}
\usepackage{mathtools}
\usepackage{minibox}
\usepackage{minted}  % Syntax highlighting using pygmentize
\usepackage{multicol}
\usepackage{pbox}
\usepackage{tabularx}
\usepackage{times}
\usepackage[normalem]{ulem}
\usepackage{wasysym}
\usepackage{xparse}
\usepackage{xspace}
\usepackage[utf8]{inputenc}
\usepackage{agda}
\usepackage{etoolbox}

\usepackage{minted}
\newminted[Haskell]{haskell}{mathescape,escapeinside=~~}
\newmintinline[Hs]{haskell}{}
\definecolor{markinlineColor}{RGB}{204,229,255}
\newmintinline[Hc]{haskell}{bgcolor=markinlineColor}


\newminted[Java]{java}{mathescape,escapeinside=~~}

\newminted[JavaScript]{javascript}{mathescape,escapeinside=~~}


\usepackage{color,colortbl,xcolor}
\usepackage{comment}

\usepackage{tikz}
\usetikzlibrary{arrows,arrows.meta}
\usetikzlibrary{calc}
\usetikzlibrary{fit}
\usetikzlibrary{overlay-beamer-styles}
\usetikzlibrary{positioning}
\usetikzlibrary{shapes,shapes.symbols}
\usetikzlibrary{tikzmark}
\usetikzlibrary{trees}

\definecolor{mc1}{RGB}{27,158,119}
\definecolor{mc2}{RGB}{217,95,2}
\definecolor{mc3}{RGB}{117,112,179}

\usepackage{pgfplots}
\usepgfplotslibrary{dateplot}

\usepackage[absolute,overlay]{textpos}  % place images with absolute position
  \setlength{\TPHorizModule}{1mm}
  \setlength{\TPVertModule}{1mm}

% Use metropolis beamer theme
\usetheme{metropolis}
\usepackage{FiraSans}  % Really nice font for metro theme
\setbeamercolor{progress bar}{fg=darkgray}

\setbeamertemplate{itemize item}{$\bullet$}
\setbeamertemplate{navigation symbols}{}
\setbeamertemplate{footline}[frame number]

% Define block styles
\tikzstyle{decision} = [diamond, draw, fill=blue!20, text width=4.5em,
  text badly centered, node distance=3cm, inner sep=0pt]
\tikzstyle{block} = [rectangle, draw, fill=mDarkTeal, font=\footnotesize,
  text=white, text width=4em, text centered, rounded corners, minimum height=3em]
\tikzstyle{line} = [draw, -latex']
\tikzstyle{cloud} = [draw, ellipse,fill=mDarkTeal!20, node distance=2.5cm,
  font=\footnotesize, minimum height=2em]

% Stuff for the title page
\title{\lecturetitle}
\author{Jesper Cockx}
\institute{Technical University Delft}
%\titlegraphic{\includegraphics[height=2cm]{TUDelft.png}}


% `to' (destination) node in code
\def\soToNode#1{\node (#1) at ($(pic cs:#1) + (-3pt,0)$) {}}
% `from' (comment) node
\DeclareDocumentCommand{\soFromNode}{ m m m }{\node (_#1) [#3 #1] {#2}}
% arrow connecting `from' and `to'
\DeclareDocumentCommand{\soArrow}{ m m }{\draw [#2] (_#1) -> (#1)}
% #1             #2                   #3                 #4                #5
% tikzmark name, text in description, where description, arrow attributes, arguments to tikzpicture
\DeclareDocumentCommand{\soComment}{ m m O{below=4mm of} O{black,very thick} O{} }{%
\begin{tikzpicture}[remember picture, ->, >=stealth, overlay, red, very thick, align=left,#5]
  \soToNode{#1};
  \soFromNode{#1}{#2}{#3};
  \soArrow{#1}{#4};
\end{tikzpicture}
}

\tikzset{var/.style={draw,rectangle,blue,very thick,minimum size=30}}
\tikzset{obj/.style={draw,ellipse,minimum width=30,thick}}

\tikzset{cls/.style={draw,ellipse,minimum width=70,very thick}}
\tikzset{int/.style={draw,ellipse,minimum width=70,very thick, dotted}}

% http://tex.stackexchange.com/questions/100313/a-problem-with-beamer-tikz-and-fonts
\let\next\relax  % See URL above for why this is needed with TikZ + Beamer


\setbeamertemplate{section in toc}[square]

\makeatletter
\patchcmd{\beamer@sectionintoc}{\vskip1.5em}{\vskip0.5em}{}{}
\makeatother

\AtBeginSection[]
{
  \begin{frame}<beamer>[plain,noframenumbering]{Outline}
    \tableofcontents[currentsection,hideallsubsections]
  \end{frame}
}

\makeatletter
\newlength\beamerleftmargin
\setlength\beamerleftmargin{\Gm@lmargin}
\makeatother

\renewcommand{\arraycolsep}{3pt}

\newdimen\mathindent
\mathindent=2ex


\usepackage{newunicodechar}
\newunicodechar{→}{\ensuremath{\rightarrow}}
\newunicodechar{←}{\ensuremath{\leftarrow}}
\newunicodechar{×}{\ensuremath{\times}}
\newunicodechar{λ}{\ensuremath{\lambda}}
\newunicodechar{∀}{\ensuremath{\forall}}
\newunicodechar{Π}{\ensuremath{\Pi}}
\newunicodechar{Σ}{\ensuremath{\Sigma}}
\newunicodechar{≡}{\ensuremath{\equiv}}
\newunicodechar{≅}{\ensuremath{\cong}}
\newunicodechar{≐}{\ensuremath{\doteq}}
\newunicodechar{∈}{\ensuremath{\in}}
\newunicodechar{∧}{\ensuremath{\land}}
\newunicodechar{∨}{\ensuremath{\lor}}
\newunicodechar{⊤}{\ensuremath{\top}}
\newunicodechar{⊥}{\ensuremath{\bot}}
\newunicodechar{⊔}{\ensuremath{\sqcup}}
\newunicodechar{∷}{\ensuremath{{::}}}
\newunicodechar{ℓ}{\ensuremath{\ell}}
\newunicodechar{₀}{\ensuremath{{_0}}}
\newunicodechar{₁}{\ensuremath{{_1}}}
\newunicodechar{₂}{\ensuremath{{_2}}}
\newunicodechar{₃}{\ensuremath{{_3}}}
\newunicodechar{₄}{\ensuremath{{_4}}}
\newunicodechar{₅}{\ensuremath{{_5}}}
\newunicodechar{₆}{\ensuremath{{_6}}}
\newunicodechar{₇}{\ensuremath{{_7}}}
\newunicodechar{₈}{\ensuremath{{_8}}}
\newunicodechar{₉}{\ensuremath{{_9}}}
\newunicodechar{⟨}{\ensuremath{{\langle}}}
\newunicodechar{⟩}{\ensuremath{{\rangle}}}
\newunicodechar{̧}{\c}
\newunicodechar{∘}{\ensuremath{\circ}}
\newunicodechar{≤}{\ensuremath{\leq}}
\newunicodechar{⋯}{\ensuremath{\cdots}}
\newunicodechar{ℕ}{\ensuremath{\mathbb{N}}}
\newunicodechar{∎}{\ensuremath{\blacksquare}}
\newunicodechar{∞}{\ensuremath{\infty}}
\newunicodechar{⊎}{\ensuremath{\uplus}}
\newunicodechar{¬}{\ensuremath{\lnot}}
\newunicodechar{⇓}{\ensuremath{\Downarrow}}
\newunicodechar{≟}{\ensuremath{\stackrel{?}{=}}}
\newunicodechar{ν}{\ensuremath{\nu}}
\newunicodechar{δ}{\ensuremath{\delta}}
\newunicodechar{∅}{\ensuremath{\emptyset}}
\newunicodechar{ε}{\ensuremath{\varepsilon}}
\newunicodechar{∋}{\ensuremath{\ni}}
\newunicodechar{∩}{\ensuremath{\cap}}
\newunicodechar{∪}{\ensuremath{\cup}}
\newunicodechar{·}{\ensuremath{\cdot}}

\newcommand\var[1]{\mathit{#1}}
\newcommand\prim[1]{{\AgdaPrimitive{#1}}}
\newcommand\ty[1]{{{\prim{Set}}_{#1}}}
\newcommand\fun[1]{{\AgdaFunction{#1}}}
\newcommand\data[1]{{\AgdaFunction{#1}}}
\newcommand\con[1]{{\AgdaInductiveConstructor{#1}}}
\newcommand\field[1]{{\AgdaField{#1}}}
\newcommand\keyw[1]{{\AgdaKeyword{#1}}}
\newcommand\lit[1]{{\AgdaNumber{#1}}}
\newcommand\level{\fun{Level}}
\newcommand\ite[3]{\fun{if}\  #1\  \fun{then}\  #2\  \fun{else}\  #3}


\newcommand\Nat{\data{ℕ}}
\newcommand\zero{\con{zero}}
\newcommand\suc{\con{suc}}
\newcommand\Bool{\data{Bool}}
\newcommand\true{\con{true}}
\newcommand\false{\con{false}}
\newcommand\List{\data{List}}
\renewcommand\Vec{\data{Vec}}
\newcommand\nil{\con{[]}}
\newcommand\cons{\con{::}}
\newcommand\Fin{\data{Fin}}
\renewcommand\prod{\data{×}}
\newcommand\sigmatype{\data{Σ}}
\newcommand\toptype{\data{⊤}}
\newcommand\bottomtype{\data{⊥}}
\newcommand\Id{\data{≡}}
\newcommand\refl{\con{refl}}

\newcommand\prf[1]{\colorbox{gray!30}{#1}}
\newcommand{\ru}[2]{\dfrac{\begin{array}{@{}c@{}}#1\end{array}}{\begin{array}{@{}c@{}}#2\end{array}}}



%%% Local Variables:
%%% mode: latex -shell-escape
%%% TeX-master: t
%%% End:


\def\lecturetitle{Coinductive Programming and Proving in Agda}
\newcommand\courseyear{2026}
\subtitle{Lecture 2: Coinductive proving in Agda}
\date{21 January 2026}

\begin{document}

\maketitle

\begin{frame}[plain,noframenumbering]{Lecture plan}
  \tableofcontents[hideallsubsections]
\end{frame}

\begin{code}[hide]
{-# OPTIONS --allow-unsolved-metas --guardedness --sized-types --cubical -WnoUnsupportedIndexedMatch #-}
open import lec1-slides
\end{code}



\section{Curry-Howard}

\begin{frame}{Curry-Howard for propositional logic}
We can interpret logical propositions as the \alert{types} of all their possible proofs.
\begin{center} \small
\begin{tabular}{r>{\columncolor{gray!30}}cl}
\textbf{Propositional logic} & & \textbf{Type system} \\
conjunction & $P\ \prod\ Q$ & pair type \\
disjunction & $\data{Either}\ P\ Q$ & either type \\
implication & $P \rightarrow Q$ & function type \\
truth & $\toptype$ & unit type \\
falsity & $\bottomtype$ & empty type \\
\end{tabular}
\end{center}
To prove a proposition, we just implement a function of the corresponding type!
\end{frame}


\begin{code}[hide]
modusPonens : {P Q : Set} → (P → Q) → P → Q
modusPonens f x = f x
\end{code}

\begin{code}[hide]
cases : {P Q R : Set}
      → Either P Q → (P → R) → (Q → R) → R
cases (left x)   f g = f x
cases (right y)  f g = g y
\end{code}

\begin{code}[hide]
¬_ : Set → Set
¬ A = A → ⊥
\end{code}



\begin{frame}{Constructive logic}
In classical logic we can prove certain `non-constructive' statements:
\begin{itemize}
\item $P \lor (\lnot P)$ \hfill (excluded middle)
\item $\lnot \lnot P \Rightarrow P$ \hfill (double negation elimination)
\end{itemize}
However, Agda uses a \alert{constructive logic}: a proof of $A \lor B$ gives us a \alert{decision procedure}
to tell whether $A$ or $B$ holds.

When $P$ is unknown, it's impossible to decide whether $P$ or $\lnot P$ holds,
so the excluded middle is \alert{unprovable} in Agda.
\end{frame}

\begin{frame}{From classical to constructive logic}
Consider the proposition $P$ (``$P$ is true'') vs.~ $\lnot\lnot P$ (``It would be absurd if $P$ were false'').

Classical logic can't tell the difference between the two, but constructive logic can.

\textbf{Theorem} (G\"odel and Gentzen). $P$ is provable in classical logic if and only if $\lnot\lnot P$ is provable in constructive logic.

\textbf{Exercise.} Prove that the double negation of the excluded middle holds in Agda.
\end{frame}


\begin{frame}{Defining predicates}
  We can define a predicate on type $A$ as a  \alert{dependent type} with base type $A$. For example:
\begin{code}
data IsEven : ℕ → Set where
  e-zero : IsEven zero
  e-suc2 : IsEven n → IsEven (suc (suc n))

two-is-even : IsEven 2
two-is-even = e-suc2 e-zero

five-is-not-even : IsEven 5 → ⊥
five-is-not-even (e-suc2 (e-suc2 ()))
\end{code}
\end{frame}


\begin{frame}{Induction in Agda}
In Agda, a \alert{proof by induction} is simply a function using pattern matching and recursion:
\begin{code}
double : ℕ → ℕ
double zero     = zero
double (suc m)  = suc (suc (double m))

double-even : (n : ℕ) → IsEven (double n)
double-even zero     = e-zero
double-even (suc m)  = e-suc2 (double-even m)
\end{code}
\end{frame}

\begin{frame}{Proving things about programs}
  \textbf{General rule of thumb:} A proof about a function often follows the same structure as that function:
  \begin{itemize}
    \item To prove something about a function by pattern matching, the proof will also use pattern matching (= \alert{proof by cases})
    \item To prove something about a recursive function, the proof will also be recursive (= \alert{proof by induction})
  \end{itemize}
\end{frame}

\section{Properties of coinductive types}

\begin{frame}{Finite colists}
\data{Finite} is an \alert{inductive} predicate on a (mixed) coinductive type:
\begin{code}[hide]
module FiniteInfinite where
  open Guardedness
  variable x xs : A
\end{code}
\begin{code}
  data Finite {A : Set} : Colist A → Set where
    [] : Finite []
    -∷_ : Finite (xs .force) → Finite (x ∷ xs)

  fromListFin : (xs : List A) → Finite (fromList xs)
  fromListFin [] = []
  fromListFin (x ∷ xs) = -∷ fromListFin xs
\end{code}
\end{frame}

\begin{frame}{Converting back to a list}
We can convert a finite colist back to a list:
\begin{code}
  toList : (xs : Colist A) → Finite xs → List A
  toList []        []        = []
  toList (x ∷ xs)  (-∷ fin)  = x ∷ toList (xs .force) fin
\end{code}
\textbf{Question.} Is this function using induction or coinduction?
\end{frame}

\begin{frame}{Infinite colists}
\data{Infinite} is a \alert{coinductive} predicate on colists:
\begin{code}
  mutual
    data Infinite {A : Set} : Colist A → Set where
      -∷_ : Infinite' xs → Infinite (x ∷ xs)

    record Infinite' (xs : Colist' A) : Set where
      coinductive
      field force : Infinite (xs .force)
  open Infinite' public
\end{code}
\textbf{Exercise.} Prove that \fun{fromStream} always produces an infinite colist.
\begin{code}[hide]
  fromStreamInf : (xs : Stream A) → Infinite (fromStream xs)
  fromStreamInf xs = -∷ λ where .force → fromStreamInf (xs .tailS)
\end{code}
\end{frame}

\begin{frame}{Converting back to a stream}
\textbf{Exercise.} Implement the following function:
\begin{code}
  toStream :
    (xs : Colist A) → Infinite xs → Stream A
\end{code}
\begin{code}[hide]
  toStream (x ∷ xs) (-∷ inf) .headS = x
  toStream (x ∷ xs) (-∷ inf) .tailS = toStream (xs .force) (inf .force)
\end{code}
\textbf{Question.} What should the function do in the case of an empty colist?

\textbf{Question.} Is this function using induction or coinduction?
\end{frame}

\begin{frame}{Finite or infinite?}
\textbf{Question.} Can we prove the following?
\begin{code}
  finite-or-infinite : (xs : Colist A) →
    Finite xs ⊎ Infinite xs
\end{code} \pause
\textbf{Answer.} No, but we can prove this instead:
\begin{code}
  infinite-not-finite : Infinite xs → ¬ (Finite xs)
  not-finite-infinite : ¬ (Finite xs) → Infinite xs
\end{code}
where $¬~A = A → ⊥$. \textbf{Exercise.} Do it!
\begin{code}[hide]
  finite-or-infinite xs = {!   !}
  infinite-not-finite = {!   !}
  not-finite-infinite = {!   !}
\end{code}
\end{frame}

\section{The identity type}

\begin{frame}{The identity type}
The \alert{identity type} $x~\Id~y$ says $x$ and $y$ are equal:
\begin{code}[hide]
variable x y z : A
module InductiveIdentity where
\end{code}
\begin{code}
  data _≡_ {A : Set} : A → A → Set where
    refl : x ≡ x
\end{code}
\begin{code}[hide]
  infix 4 _≡_
\end{code}
The constructor $\refl$ proves that two terms are equal if they have the same normal form:
\begin{code}
  one-plus-one : 1 + 1 ≡ 2
  one-plus-one = refl
\end{code}
\end{frame}


\begin{frame}{Quiz question}
\textbf{Question.} What is the type of the Agda expression $\lambda~b \to (b~\Id~\true)$?
\begin{enumerate}
\item $\data{Bool} \to \data{Bool}$
\item $\data{Bool} \to \data{Set}$
\item $(b : \data{Bool}) \to b~\Id~\true$
\item It is not a well-typed expression
\end{enumerate}
\end{frame}

\begin{frame}{Application of the identity type: \\ Writing test cases}
One use case of the identity type is for writing test cases:
\begin{code}
  test₁ : length (42 ∷ []) ≡ 1
  test₁ = refl

  test₂ : length (map (1 +_) (0 ∷ 1 ∷ 2 ∷ [])) ≡ 3
  test₂ = refl
\end{code}
The test cases are run \alert{each time the file is loaded}!
\end{frame}

\begin{frame}{Proving correctness of functions}
We can use the identity type to prove the correctness of functional programs.

\textbf{Example.}
\begin{code}[hide]
  module NotNotSimple where
\end{code}
\begin{code}
    not-not : (b : Bool) → not (not b) ≡ b
    not-not true   = refl
    not-not false  = refl
\end{code}
\end{frame}


\begin{frame}{Pattern matching on $\texttt{refl}$}
  If we have a proof of $x~\Id~y$ as input, we can \alert{pattern match} on the constructor $\refl$
  to show Agda that $x$ and $y$ are equal:
\begin{code}
  castVec : m ≡ n → Vec A m → Vec A n
  castVec refl xs = xs
\end{code}
When you pattern match on $\refl$, Agda applies \alert{unification} to the two sides of the equality.
\end{frame}

\begin{frame}{Properties of equality}
\begin{code}
  sym : x ≡ y → y ≡ x
  sym refl = refl

  trans : x ≡ y → y ≡ z → x ≡ z
  trans refl refl = refl

  cong : (f : A → B) → x ≡ y → f x ≡ f y
  cong f refl = refl
\end{code}
\end{frame}

\begin{frame}{From lists to colists and back}
\textbf{Exercise.} Prove that converting a list to a colist and back is the identity:
\begin{code}[hide]
  module FromListInv where
    open Guardedness
    open FiniteInfinite
\end{code}
\begin{code}
    fromListInv : (xs : List A)
      → toList (fromList xs) (fromListFin xs) ≡ xs
\end{code}
\begin{code}[hide]
    fromListInv [] = refl
    fromListInv (x ∷ xs) = cong (x ∷_) (fromListInv xs)
\end{code}
Can we prove the same about \fun{fromStream}?
\end{frame}

\section{Equational reasoning}

\begin{frame}[fragile]{Equational reasoning}
We can write more readable identity proofs by using \alert{equational reasoning operators}:
\begin{code}
  _≡⟨_⟩_ : (x : A) → x ≡ y → y ≡ z → x ≡ z
  x ≡⟨ refl ⟩ q = q

  _≡⟨⟩_ : (x : A) → x ≡ y → x ≡ y
  x ≡⟨⟩ q = x ≡⟨ refl ⟩ q

  _∎ : (x : A) → x ≡ x
  x ∎ = refl
\end{code}
\begin{code}[hide]
  infix   3  _∎
  infixr  2  _≡⟨_⟩_
  infixr  2  _≡⟨⟩_
\end{code}
\end{frame}

\begin{code}[hide]
  module ReasoningExamples where
\end{code}

\begin{frame}{Equational reasoning example}
\begin{code}
    reverse : List A → List A
    reverse []        = []
    reverse (x ∷ xs)  = reverse xs ++ (x ∷ [])

    reverse-singleton : reverse (x ∷ []) ≡ x ∷ []
    reverse-singleton {x = x} =
      reverse (x ∷ [])        ≡⟨⟩
      reverse [] ++ (x ∷ [])  ≡⟨⟩
      [] ++ (x ∷ [])          ≡⟨⟩
      (x ∷ [])                ∎
\end{code}
\end{frame}



\begin{frame}{Equational reasoning + induction}
\begin{code}
    add-n-zero : (n : ℕ) → n + zero ≡ n
    add-n-zero zero     = refl
    add-n-zero (suc n)  =
      (suc n) + zero   ≡⟨⟩
      suc (n + zero)   ≡⟨ cong suc (add-n-zero n) ⟩
      suc n            ∎
\end{code}
Here we have to provide an \alert{explicit proof} that $\suc\ (n~\fun{+}~\zero) = \suc~n$, using the IH.

\textbf{Exercise.} Prove that $\var{xs}~\fun{++}~\con{[]} = \var{xs}$.

\textbf{Exercise.} Prove associativity of \fun{++}.
\end{frame}


\begin{frame}{Example 1: functor laws for $\texttt{List}$}
The first functor law for lists:
\begin{code}
    map-id : {A : Set} (xs : List A) → map id xs ≡ xs
    map-id []        = refl
    map-id (x ∷ xs)  =
      map id (x ∷ xs)   ≡⟨⟩
      id x ∷ map id xs  ≡⟨⟩
      x ∷ map id xs     ≡⟨ cong (x ∷_) (map-id xs) ⟩
      x ∷ xs            ∎
\end{code}
\end{frame}

\begin{frame}{Exercise}
Prove the second functor law for $\List$.

First, we need to define function composition:\footnote{Unicode input for $∘$: \texttt{\textbackslash{}circ}}
\begin{code}
    _∘_ : {A B C : Set} →
      (B → C) → (A → B) → (A → C)
    f ∘ g = λ x → f (g x)
\end{code}
Now we can prove that $\fun{map}\ (f\ \fun{∘}\ g)\ x = (\fun{map}\ f\ \fun{∘}\ \fun{map}\ g)\ x$.
\end{frame}

\begin{frame}{Example 2: verifying optimizations}
A faster version of $\fun{reverse}$ in $O(n)$:
\begin{code}
    reverse-acc : List A → List A → List A
    reverse-acc []        ys = ys
    reverse-acc (x ∷ xs) ys = reverse-acc xs (x ∷ ys)

    reverse' : List A → List A
    reverse' xs = reverse-acc xs []
\end{code}
\end{frame}

\begin{frame}{Equivalence of $\texttt{reverse}$ and $\texttt{reverse'}$} \small

\begin{code}[hide]
    postulate
      append-[] : (xs : List A) → xs ++ [] ≡ xs
      append-assoc : (xs ys zs : List A)
                    → (xs ++ ys) ++ zs ≡ xs ++ (ys ++ zs)
\end{code}
\begin{code}
    reverse-acc-lemma : (xs ys : List A)
      → reverse-acc xs ys ≡ reverse xs ++ ys
    reverse-acc-lemma [] ys = refl
    reverse-acc-lemma (x ∷ xs) ys =
      reverse-acc (x ∷ xs) ys         ≡⟨⟩
      reverse-acc xs (x ∷ ys)
        ≡⟨ reverse-acc-lemma xs (x ∷ ys) ⟩
      reverse xs ++ (x ∷ ys)
        ≡⟨ sym (append-assoc (reverse xs) (x ∷ []) ys) ⟩
      (reverse xs ++ (x ∷ [])) ++ ys  ≡⟨⟩
      reverse (x ∷ xs) ++ ys          ∎
\end{code}
\textbf{Exercise.} Use this to prove that \fun{reverse} and \fun{reverse'} are equivalent.
\begin{code}[hide]
    reverse'-reverse : (xs : List A) → reverse' xs ≡ reverse xs
    reverse'-reverse xs =
      reverse' xs         ≡⟨⟩
      reverse-acc xs []   ≡⟨ reverse-acc-lemma xs [] ⟩
      reverse xs ++ []    ≡⟨ append-[] (reverse xs) ⟩
      reverse xs          ∎
\end{code}

\end{frame}

\begin{comment}

\begin{frame}{Example 3: Verifying a compiler}
\begin{code}[hide]
  open ReasoningExamples
\end{code}
A simple expression language:\footnote{Known as ``Hutton's Razor''}
\begin{code}
  data Expr : Set where
    valE  : ℕ → Expr
    addE  : Expr → Expr → Expr

  [2+3]+4 : Expr
  [2+3]+4 = addE (addE (valE 2) (valE 3)) (valE 4)

  eval : Expr → ℕ
  eval (valE x)      = x
  eval (addE e1 e2)  = eval e1 + eval e2
\end{code}
\end{frame}

\begin{frame}{Evaluating expressions using a stack}
\begin{code}
  data Op : Set where
    PUSH  : ℕ → Op
    ADD   : Op

  Stack  = List ℕ
  Code   = List Op

  -- Example code for (2 + 3) + 4
  code : Code
  code = PUSH 2 ∷ PUSH 3 ∷ ADD
          ∷ PUSH 4 ∷ ADD ∷ []
\end{code}
\end{frame}

\begin{frame}{Executing compiled code}
Given a list of instructions and an initial stack, we can execute the code:
\begin{code}
  exec : Code → Stack → Stack
  exec []             s              = s
  exec (PUSH x ∷ c)  s              = exec c (x ∷ s)
  exec (ADD    ∷ c)  (m ∷ n ∷ s)  = exec c (n + m ∷ s)
  exec (ADD    ∷ c)  _              = []
\end{code}
\end{frame}

\begin{frame}{Compiling expressions}
Our compiler uses an \alert{accumulator} for the already generated code:
\begin{code}
  comp' : Expr → Code → Code
  comp' (valE x)      c  = PUSH x ∷ c
  comp' (addE e1 e2)  c  =
    comp' e1 (comp' e2 (ADD ∷ c))

  comp : Expr → Code
  comp e = comp' e []
\end{code}
\end{frame}

\begin{frame}{Proving correctness of \texttt{comp}}
\begin{code}
  comp-exec-eval : (e : Expr)
    → exec (comp e) [] ≡ eval e ∷ []
  comp-exec-eval e =
    exec (comp e) []       ≡⟨ comp'-exec-eval e [] [] ⟩
    exec [] (eval e ∷ [])  ≡⟨⟩
    eval e ∷ []            ∎
\end{code}
\begin{code}[hide]
    where
\end{code}
\end{frame}

\begin{frame}{Proving correctness of \texttt{comp'} (\texttt{valE} case)}
To prove correctness of \fun{comp'}, we need to generalize the statement to an arbitrary stack:
\begin{code}
    comp'-exec-eval : (e : Expr) (s : Stack) (c : Code)
      → exec (comp' e c) s ≡ exec c (eval e ∷ s)
    comp'-exec-eval (valE x) s c =
      exec (comp' (valE x) c) s     ≡⟨⟩
      exec (PUSH x ∷ c) s           ≡⟨⟩
      exec c (x ∷ s)                ≡⟨⟩
      exec c (eval (valE x) ∷ s)    ∎
\end{code}
\end{frame}

\begin{frame}{Proving correctness of \texttt{comp'} (\texttt{addE} case)}
\begin{code}
    comp'-exec-eval (addE e1 e2) s c =
      exec (comp' (addE e1 e2) c) s             ≡⟨⟩
      exec (comp' e1 (comp' e2 (ADD ∷ c))) s
        ≡⟨ comp'-exec-eval e1 s (comp' e2 (ADD ∷ c)) ⟩
      exec (comp' e2 (ADD ∷ c)) (eval e1 ∷ s)
        ≡⟨ comp'-exec-eval e2 (eval e1 ∷ s) (ADD ∷ c) ⟩
      exec (ADD ∷ c) (eval e2 ∷ eval e1 ∷ s)    ≡⟨⟩
      exec c (eval e1 + eval e2 ∷ s)            ≡⟨⟩
      exec c (eval (addE e1 e2) ∷ s)            ∎
\end{code}
\end{frame}

\end{comment}

\section{Bisimulation}


\begin{code}[hide]
  module GuardedBisimulation where
    open Guardedness
\end{code}

\begin{frame}{Proving an equality between streams}
Let's prove an equality on streams:
\begin{code}[hide]
    module WrongTakeDrop where
\end{code}
\begin{code}
      takeDropS  : (n : ℕ) (s : Stream A)
                → takeS n s ++S dropS n s ≡ s
      takeDropS zero s = refl
      takeDropS (suc n) s = {! cong (s .headS ∷S_) ? !}
\end{code}
{Error: \color{red}\underline{s .headS ∷S \_y\_296 != s}}
\end{frame}

\begin{frame}{Bisimulation of streams}
Streams are coinductive, but the identity type is \alert{inductive}:
Agda needs to `see' both sides are equal in a finite number of steps.

We need a \alert{coinductive} relation instead:
\begin{code}
    record _~_ {A : Set} (s1 s2 : Stream A) : Set where
      coinductive
      field
        headS  :  s1  .headS  ≡  s2  .headS
        tailS  :  s1  .tailS  ~  s2  .tailS
    open _~_ public
\end{code}
\end{frame}

\begin{frame}{Proving bisimulation of streams}
Proving bisimulation is just defining a coinductive value of the right type:
\begin{code}
    refl~ : (s : Stream A) → s ~ s
    refl~ s .headS = refl
    refl~ s .tailS = refl~ (s .tailS)

    takeDropS  : (n : ℕ) (s : Stream A)
              → (takeS n s ++S dropS n s) ~ s
    takeDropS zero     s         = refl~ s
    takeDropS (suc n)  s .headS  = refl
    takeDropS (suc n)  s .tailS  = takeDropS n (s .tailS)
\end{code}
\end{frame}

\begin{frame}{From streams to colists and back}
\textbf{Exercise.} Prove that converting a stream to a colist and back results in a stream that is bisimilar to the original one:
\begin{code}[hide]
    module FromStreamInv where
      open FiniteInfinite
\end{code}
\begin{code}
      fromStreamInv : (xs : Stream A)
        → toStream (fromStream xs) (fromStreamInf xs)
          ~ xs
\end{code}
\begin{code}[hide]
      fromStreamInv xs .headS = refl
      fromStreamInv xs .tailS = fromStreamInv (xs .tailS)
\end{code}
\end{frame}


\section{Cubical bisimulation}

\begin{frame}{Cubical Agda}

Cubical Agda is an extension of Agda with primitives from \alert{cubical type theory},
a version of homotopy type theory (HoTT).

In particular, it provides the cubical path type,
a version of \alert{observational equality}.

Cubical Agda also provides other primitives like \fun{glue} and \fun{hcomp},
which are needed to prove the principle of \alert{univalence}.
\end{frame}

\begin{frame}{The cubical path type}
The cubical \alert{interval} type \data{I} is a type with two elements \fun{i0} and \fun{i1}
that \emph{cannot be distinguished} from inside Agda.

The cubical path type $\data{Path}~A~x~y$ (sometimes also written $x~\Id~y$) is the type of
functions $f : I → A$ such that $f~\fun{i0} = x$ and $f~\fun{i1} = y$.
\begin{code}[hide]
module CubicalIdentity where
  open import Cubical.Foundations.Prelude using (I; Path; _≡_; i0; i1)
  open Guardedness
\end{code}
\end{frame}

\begin{frame}{Properties of the path type}
\begin{code}
  reflP : {x : A} → Path A x x
  reflP {x = x} i = x

  congP : (f : A → B) → Path A x y → Path B (f x) (f y)
  congP f p i = f (p i)
\end{code}
\fun{symP} and \fun{transP} need additional cubical primitives, let's not worry about it for now.

In fact, we can prove that $\data{Path}~A~x~y$ is \alert{isomorphic} to the inductive identity type!
\end{frame}


\begin{frame}{Functional extensionality}
\alert{Functional extensionality} states that two functions are equal if they give equal outputs on every input.

For inductive identity, functional extensionality is consistent but \alert{unprovable}.

For the cubical path type, it is trivial:
\begin{code}
  funExt  : {f g : A → B}
          → ((x : A) → Path B (f x) (g x))
          → Path (A → B) f g
  funExt h i x = h x i
\end{code}
\end{frame}

\begin{frame}{Cubical bisimulation}
The cubical path type serves as a general bisimulation relation for any coinductive type:
\begin{code}
  takeDropS  : (n : ℕ) (s : Stream A)
    → Path (Stream A) (takeS n s ++S dropS n s) s
  takeDropS zero     s i         = s
  takeDropS (suc n)  s i .headS  = s .headS
  takeDropS (suc n)  s i .tailS  =
    takeDropS n (s .tailS) i
\end{code}
\textbf{Exercise.} Prove that bisimilarity of streams implies path equality.
\end{frame}

\section*{Conclusion}

\begin{frame}{Next time: coinduction case studies}
\begin{itemize}
\item The delay monad
\item Stream processors
\item Formal languages
\item Wander types
\item Coinductive graphs (?)
\end{itemize}
\end{frame}

\end{document}






%%% Local Variables:
%%% mode: latex
%%% TeX-master: t
%%% LaTeX-command: "latex -shell-escape"
%%% TeX-command-extra-options: "-shell-escape"
%%% End:
